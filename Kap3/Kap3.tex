\chapter{Estudios del Cotransportador vSGLT}


\section{Transportadores}
La membrana celular al ser hidrof\'{o}bica permite protegerse de la regi\'{o}n extracelular, sin embargo, ella necesita ingresar y expulsar todos los compuestos necesarios para realizar su fisiolog\'{i}a OJO. La membranana celular tiene prote\'{i}nas que permiten el ingreso y la expulsi\'{o}n de estos compuestos. Entre los tipos de prote\'{i}nas se encuentran  los poros, las bombas, los transportadores y los canales. \\

Los transportadores se clasifican, de acuerdo al sistema de clasificaci\'{o}n de transportadores \cite{Nelson2011}, en dos categor\'{i}as principales de las cuales se desprenden otras subcategor\'{i}as :
\begin{enumerate}
 \item \textbf{Portadores}:
 \begin{enumerate}
 \item[1.] Transportadores activos primarios
 \item[2.a] Transportadores activos secundarios
  \begin{enumerate}
 \item[a)] Simportadores
 \item[b)] Antiportadores
 \end{enumerate}
 \item[2.b] Uniportadores
 \end{enumerate}
 \item  \textbf{Canales i\'{o}nicos}: Los canales se diferencian de los portadores en la raz\'{o}n a la que transportan los iones, que es de $10^6$ iones/segundo  (muy alta) as\'{i} como en que los canales no necesitan energ\'{i}a metab\'{o}lica para transportar los iones.
\end{enumerate}


Para visualizar las prote\'{i}nas se puede usar el microscopio de fuerza at\'{o}mico
\subsection{Cotransportadores}
Hacia la d\'{e}cada de los a\~{n}os sesenta Robert Crane, ver \cite{Hamilton2013}, estableci\'{o} una relaci\'{o}n acoplada o de \textit{cotransporte} entre el ion de sodio y la glucosa los cuales son absorbidos por el intestino delgado. El conocimiento de este mecanismo ha permitido realizar el tratamiento de la diarrea y del c\'{o}lera mediante la rehidrataci\'{o}n oral. La hip\'{o}tesis del cotransporte que ha sido numerosamente validada y ha sido una piedra angular para el entendimiento de el metabolismo de los carbohidratos, claves en la energ\'{e}tica celular. La hip\'{o}tesis del cotransporte tambi\'{e}n se ha extendido a otros organismos vivos, con la diferencia de que el acoplamiento del sodio se puede dar con cualquier otro soluto org\'{a}nico \cite{Faham2008}.\\
\section{Algunas Familias Proteicas y enrollamiento LeuT (LeuT fold)}
\section{Co-transportador vSGLT}
El cotransportador de Na+/Galactosa presente en la proteobacteria \textit{Vibrio Parparahaemolyticus } denotado como vSGLT, es un simportador perteneciente a la familia de simportadores de Na$+$-soluto SSS.\\
La caracterizaci\'{o}n molecular del compuesto se realiz\'{o} por primera vez hacia el a\~{n}o 2000 ver \cite{Turk2000}, mientras que la determinaci\'{o}n de su estructura fue posible hacia el a\~{n}o 2008 \cite{Faham2008}. En dichos estudios se establece que vSGLT es similar en la estructura primaria y terciaria a otros transportadores de la familia SLC5 como por ejemplo NIS, SGLT1, SGLT2.
\section{Estudios actuales del Co-transportador vSGLT}
