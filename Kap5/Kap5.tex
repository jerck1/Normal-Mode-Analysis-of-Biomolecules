\chapter{Conclusiones y Recomendaciones}\label{ch:5}
\addcontentsline{toc}{chapter}{\numberline{}Conclusiones y Recomendaciones}
%\section{Conclusiones}
Para distancias de corte entre $8\AA$ y $9\AA$, se observa que la prote\'{i}na  es menos flexible cuando est\'{a} sin unirse a los sustratos (color rojo) desde el segmento TM2 hasta el final del segmento TM14. La excepci\'{o}n es el pico presente en el residuo W543 (ver tabla \ref{tab:flu2} del anexo \ref{AnexoB}) que presenta mucha m\'{a}s flexibilidad. Esto se debe a que el segmento TM14 est\'{a} lejos tanto del sitio activo de la prote\'{i}na como de la segunda subunidad como se observa en la figura \ref{fig:complejo} donde TM14 es la $\alpha$ h\'{e}lice de color rojo.\\

Los picos correspondientes a los residuos S209 y F210 tambi\'{e}n muestran cambios significativos de la fluctuaciones ms antes y despu\'{e}s de unirse alguno de los sustratos.
%\section{Recomendaciones}
%Se presentan como una serie de aspectos que se podr\'{\i}an realizar en un futuro para emprender investigaciones similares o fortalecer la investigaci\'{o}n realizada. Deben contemplar las perspectivas de la investigaci\'{o}n, las cuales son sugerencias, proyecciones o alternativas que se presentan para modificar, cambiar o incidir sobre una situaci\'{o}n espec\'{\i}fica o una problem\'{a}tica encontrada. Pueden presentarse como un texto con caracter\'{\i}sticas argumentativas, resultado de una reflexi\'{o}n acerca de la tesis o trabajo de investigaci\'{o}n.\\