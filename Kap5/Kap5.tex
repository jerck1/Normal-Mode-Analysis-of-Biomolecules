\chapter{Conclusiones y Recomendaciones}\label{ch:5}
\addcontentsline{toc}{chapter}{\numberline{}Conclusiones y Recomendaciones}
%\section{Conclusiones}
Al realizar el an\'{a}lisis de ANM con un archivo pdb se encontraron los mismos picos en todas las distancias de corte. Para archivo PDB con c\'{o}digo 3DH4 los 5 picos est\'{a}n en los residuos  S209, M318, T343, I386 y W543. Estos lugares correspondieron a sitios que conectan segmentos transmembranales, de ah\'{i} se deben sus altos valores de fluctuaciones ms. Para el archivo PDB con la primera h\'{e}lice resuelta (TM1) se encontraron otros residuos: A47,M318, F542 y G556 que se encontraban poco conectados con los dem\'{a}s residuos.\\

Para archivo PDB con c\'{o}digo 3DH4 al usar las distancias de corte entre $8\AA$ y $9\AA$, se observa que la prote\'{i}na  es menos flexible cuando est\'{a} sin unirse a los sustratos  desde el segmento TM2 hasta el final del segmento TM14. La excepci\'{o}n es el pico presente en el residuo W543 (ver tabla \ref{tab:flu2} del anexo \ref{AnexoB}) que presenta mucha m\'{a}s flexibilidad. Para las otras distancias de corte no ocurre esto.\\

Exceptuando la distancia de corte de $10\AA$, se encontraron cambios en las fluctuaciones ms en los segmentos TM3 y TM7 de la siguiente manera: Dada la flexibilidad del segmento con el cotransportador solo, el cotransportador con el sodio unido tiene m\'{a}s flexibilidad, cuando se unen tanto el sodio como la galactosa el segmento se vuelve menos flexible. Este comportamiento verifica que hay una cooperaci\'{o}n entre el sodio y la galactosa.\\

Las $\alpha$ h\'{e}lices TM1 (primera h\'{e}lice resuelta) y TM15 encontradas en la estructura cristalina de 2XQ2, contribuyen a la estabilidad de la prote\'{i}na cuando se encuentra unida al substrato y al ion, haciendo que los cambios en las fluctuaciones ms sean menores.\\

A diferencia del archivo 3DH4 del ANM previo, usando el PDB con la primera h\'{e}lice resuelta se encontraron cambios en las fluctuaciones ms en el segmento TM2. Este segmento es uno de los que participan almacenando el sodio y la galactosa en el interior del cotransportador.\\

Se encuentra que los residuos y los segmentos transmembranales que generan mayores cambios en los movimientos globales son los reportados previamente en la literatura.\\
%\section{Recomendaciones}
%Se presentan como una serie de aspectos que se podr\'{\i}an realizar en un futuro para emprender investigaciones similares o fortalecer la investigaci\'{o}n realizada. Deben contemplar las perspectivas de la investigaci\'{o}n, las cuales son sugerencias, proyecciones o alternativas que se presentan para modificar, cambiar o incidir sobre una situaci\'{o}n espec\'{\i}fica o una problem\'{a}tica encontrada. Pueden presentarse como un texto con caracter\'{\i}sticas argumentativas, resultado de una reflexi\'{o}n acerca de la tesis o trabajo de investigaci\'{o}n.\\