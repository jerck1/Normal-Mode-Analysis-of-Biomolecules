\chapter{Modelos Te\'{o}ricos}
La din\'{a}mica de una biomol\'{e}cula se determina por las ecuaciones de movimiento para cada uno de los \'{a}tomos que la constituyen. Usualmente en una biomol\'{e}cula el n\'{u}mero de mon\'{o}meros es mayor a 20, que al multiplicarlo por el n\'{u}mero de \'{a}tomos en cada mon\'{o}mero incrementa considerablemente el n\'{u}mero de ecuaciones de movimiento a resolver, de ah\'{i} que sea necesario realizar \textit{din\'{a}mica molecular} (Molecular Dynamics por sus siglas en ingl\'{e}s MD) la cual estudia mediante simulaciones computacionales el movimiento de los \'{a}tomos, de acuerdo a las interacciones que presenten.\\

Las ecuaciones de movimiento se pueden conocer a partir de los formalismos lagrangiano o hamiltoniano, en los cuales es necesario conocer los potenciales con los que interact\'{u}an los \'{a}tomos. Las soluciones a las ecuaciones de movimiento se encuentran mediante los m\'{e}todos de la din\'{a}mica molecular o los an\'{a}lisis de modos normales (Normal Mode Analysis por sus siglas en ingl\'{e}s NMA).

Los diversos modelos de potencial pueden ser tomados según la naturaleza del pol\'{i}mero a analizar, ver \cite{Amb1}. Sin embargo, al escoger el potencial  para hacer un an\'{a}lisis \textit{in silico} de la din\'{a}mica de una biomol\'{e}cula, debe tenerse en cuenta el costo computacional requerido, esto es, el tiempo de simulaci\'{o}n de la mol\'{e}cula y la exactitud en el movimiento de cada uno de los constituyentes de la mol\'{e}cula.

Las simulaciones en las que se incluyen todos los \'{a}tomos, es decir, donde se analizan los \texit{cambios locales}, se pueden simular a un orden de magnitud de los nanosegundos en una m\'{a}quina usual; mientras que las simulaciones para observar los \texit{cambios globales} pueden ser realizadas a un orden de magnitud de los microsegundos. Ver 
\subsection{Movimientos Locales}
Como caso particular se pueden tomar los potenciales usados en \cite{Amb1} y en \cite{web:Amb2}, que siguen el modelo de Amber. El modelo de Amber tiene en cuenta las contribuciones debidas a:
 \begin{itemize}
\item Interacciones intermoleculares: Son las producidas por los enlaces covalentes entre grupos de \'{a}tomos, las de valencia y las torsiones.
\item Interacciones entre pares: Lennard Jones, electrost\'{a}tico.
\end{itemize}

\subsection{Movimientos Globales}

\section{Subt\'{\i}tulos nivel 2}
Toda divisi\'{o}n o cap\'{\i}tulo, a su vez, puede subdividirse en otros niveles y s\'{o}lo se enumera hasta el tercer nivel. Los t\'{\i}tulos de segundo nivel se escriben con min\'{u}scula al margen izquierdo y sin punto final, est\'{a}n separados del texto o contenido por un interlineado posterior de 10 puntos y anterior de 20 puntos (tal y como se presenta en la plantilla).\\

\subsection{Subt\'{\i}tulos nivel 3}
De la cuarta subdivisi\'{o}n en adelante, cada nueva divisi\'{o}n o \'{\i}tem puede ser se\~{n}alada con vi\~{n}etas, conservando el mismo estilo de \'{e}sta, a lo largo de todo el documento.\\

Las subdivisiones, las vi\~{n}etas y sus textos acompa\~{n}antes deben presentarse sin sangr\'{\i}a y justificados.\\

\begin{itemize}
\item En caso que sea necesario utilizar vi\~{n}etas, use este formato (vi\~{n}etas cuadradas).
\end{itemize}