\chapter{Modelos Te\'{o}ricos}
\section{Relevancia}
Es importante estudiar los movimientos de una biomol\'{e}cula ya que como es se\~{n}alado en \cite{Lezon2009ElasticViruses} y en  \cite{Rader2006TheApplications}, la din\'{a}mica de la mol\'{e}cula vincula la estructura con la funci\'{o}n de la biomol\'{e}cula. La funci\'{o}n es el papel que desempe\~{n}a la biomol\'{e}cula y que est\'{a} intimamente relacionado con las interacciones de la biomol\'{e}cula a un ligando. La estructura de una biomol\'{e}cula tiene tiene 4 niveles de organizaci\'{o}n, denominadas \textit{estructuras primaria, secundaria, terciaria y cuaternaria}.\\

Por ejemplo, en una prote\'{i}na la estructura primaria es la secuencia u  orden de los mon\'{o}meros (amino\'{a}cidos) que la constituyen. La estructura secundaria est\'{a} determinada por los puentes de hidr\'{o}geno existentes entre los grupos amino y carboxilo de dos amino\'{a}cidos diferentes; predice la estructura tridimensional en forma local. Mientras que la estructura terciaria la da el arreglo tridimensional de cada uno de sus \'{a}tomos, espec\'{i}ficamente se determina con las coordenadas de cada uno de los \'{a}tomos. La estructura cuaternaria es el arreglo de unidades proteicas o cadenas pept\'{i}dicas formando lo que se conoce como complejo multiproteico.\\

Algunas prote\'{i}nas est\'{a}n en la clase de prote\'{i}nas de transporte que se encuentran en la membrana celular, la funci\'{o}n que cumplen es act\'{u}ar como mediadoras para el transporte de iones y sustratos, convirti\'{e}ndose este en un paso previo al metabolismo y la energ\'{e}tica celular al interior de la c\'{e}lula.\\
\section{Din\'{a}mica de una Biomol\'{e}cula}

La din\'{a}mica de una biomol\'{e}cula se determina por las ecuaciones de movimiento para cada uno de los \'{a}tomos que la constituyen. Usualmente en una biomol\'{e}cula el n\'{u}mero de mon\'{o}meros es mayor a 20, que al multiplicarlo por el n\'{u}mero de \'{a}tomos en cada mon\'{o}mero incrementa considerablemente el n\'{u}mero de ecuaciones de movimiento a resolver, de ah\'{i} que sea necesario realizar \textit{din\'{a}mica molecular} (Molecular Dynamics que por sus siglas en ingl\'{e}s es MD) la cual estudia mediante simulaciones computacionales el movimiento de los \'{a}tomos, de acuerdo a las interacciones que presenten.\\

Las ecuaciones de movimiento se pueden conocer a partir de los formalismos lagrangiano o hamiltoniano \cite{Goldstein2001ClassicalMechanics}, en los cuales es necesario conocer los potenciales con los que interact\'{u}an los \'{a}tomos. Las soluciones a las ecuaciones de movimiento se encuentran mediante los m\'{e}todos de la din\'{a}mica molecular o los an\'{a}lisis de modos normales (Normal Mode Analysis que por sus siglas en ingl\'{e}s es NMA) en los cuales se escogen los modelos de potencial.\\

Los diversos modelos de potencial pueden ser tomados seg\'{u}n la naturaleza del pol\'{i}mero a analizar, ver \cite{Lezon2009ElasticViruses}. Sin embargo, al escoger el potencial  para hacer un an\'{a}lisis \textit{in silico} de la din\'{a}mica de una biomol\'{e}cula, debe tenerse en cuenta el costo computacional requerido, esto es, el tiempo de simulaci\'{o}n de la mol\'{e}cula y la exactitud requerida en el movimiento de cada uno de los constituyentes de la mol\'{e}cula.\\

De acuerdo a los par\'{a}metros de costo y tiempo, las simulaciones de biomol\'{e}culas se pueden hacer analizando los \textit{movimientos locales} y los \textit{movimientos globales}.

\section{Movimientos Globales}

Son aqu\'{e}llas simulaciones en las que se desean conocer los \textit{cambios globales} o el aspecto general que excibe el movimiento de una biomol\'{e}cula haciendo simplificaciones, ya sea en los potenciales presentes o en el n\'{u}mero de \'{a}tomos interconectados. Este tipo de simulaciones pueden ser realizadas a un orden de magnitud de los microsegundos, lo cual facilita su uso en computadores personales, al respecto ver \cite{Gur2013GlobalPredictions.}.\\

Un conjunto de modelos que permite calcular los movimientos globales de una mol\'{e}cula son los \textit{Modelos de Redes El\'{a}sticas} (Elastic Network Models o ENM por sus siglas en ingl\'{e}s).
 Otros modelos que describen los movimientos globales son los an\'{a}lisis por componentes principales (Principal Component Analysis o PCA por sus siglas en ingl\'{e}s) y el an\'{a}lisis por modos normales est\'{a}ndar (Normal Mode Analysis o NMA por sus siglas en ingl\'{e}s).
 
\subsection{Modelos de Redes El\'{a}sticas (ENM)}
Los ENM, como la palabra \textit{el\'{a}stico} lo indica, se basan en una simplificaci\'{o}n de la energ\'{i}a potencial a una energ\'{i}a potencial el\'{a}stica, es decir de tipo Hooke.Un requisito para que sea posible hacer dicha simplificaci\'{o}n, es la de minimizar la energ\'{i}a potencial. \\

Al simplificar el potencial, la biomol\'{e}cula original se convierte en una red cuyos nodos est\'{a}n sometidos al potencial el\'{a}stico, ver figura \ref{fig:pan}. Los nodos se consideran como bloques constituyentes de la biomol\'{e}cula y no siempre los nodos coincidir\'{a}n con cada uno de los \'{a}tomos en la biomol\'{e}cula. La elecci\'{o}n del bloque constituyente depende de la compatibilidad del modelo con los datos experimentales, que se encuentra reflejado en la estabilidad de los enlaces con respecto a su posici\'{o}n de equilibrio, de tal manera que cada bloque constituyente pueda considerarse como una part\'{i}cula puntual o incluso como un cuerpo r\'{i}gido. \\
\begin{figure}
\centering%
\includegraphics[scale=0.3]{Kap2/dibujo.pdf}%
\caption{ (a) Vista exterior de un c\'{a}pside v\'{i}rico HK97 coloreado por cada cadena, todas las prote\'{i}nas son id\'{e}nticas. (b) Vista del arreglo prote\'{i}nico en una cara del c\'{a}pside. (c) Vista de la estructura secundaria de las prote\'{i}nas (d) Esquema de cada prote\'{i}na mostrando cada uno de sus \'{a}tomos, las aristas de cada cara son carbonos $\alpha$ unidos por lados (ligaduras el\'{a}sticas). Tomado de \cite{Lezon2009ElasticViruses}.} \label{fig:pan}
\end{figure}
\subsubsection{Descripci\'{o}n Mec\'{a}nica del Modelo}

Consid\'{e}rese una biomol\'{e}cula con $N$ part\'{i}culas constituyentes, el tipo de constituyente depende del modelo apropiado para la biomol\'{e}cula, por ejemplo en las prote\'{i}nas como la BPTI, ver \cite{Gur2013GlobalPredictions.}, los constituyentes son los carbonos $\alpha$ de los amino\'{a}cidos.\\

La energ\'{i}a potencial $V$ que representa las interacciones entre los constituyentes de la biomol\'{e}cula, se puede expresar alrededor de las posiciones de equilibrio $\mathbf{q_0}=\mathbf{0}$ tal como describe la teor\'{i}a de peque\~{n}as oscilaciones, ver \cite{Goldstein2001ClassicalMechanics}:
\begin{equation}
V(q)=V(\mathbf{0})+\sum_{i=1}^n\frac{\partial V}{\partial q_i}q_i+\sum_{ij}^{n}\frac{\partial^2 V }{\partial q_i\partial q_j}q_i q_j+...
\end{equation}\label{eq:1}
Donde $q_i$ son los desplazamientos con respecto a las posiciones de equilibrio, $n$ es el n\'{u}mero de posibles desplazamientos en la biomol\'{e}cula. $V(\mathbf{0})$ es el potencial en equilibrio que por conveniencia puede ser calibrado a cero: $V(\mathbf{0})=0$. \\


El sistema se encuentra alrededor del equilibrio cuando las fuerzas generalizadas se anulan, esto es:
\begin{equation}
\frac{\partial V}{\partial q_i}=0
\end{equation}\label{eq:2}
En este tipo de casos como la energ\'{i}a se minimiza, se dice que hay un equilibrio estable. Para entender esto, sup\'{o}gase que la energ\'{i}a total, que en este caso es $E=T+V$ corresponde al punto de equilibrio donde los nodos no se mueven, es decir, $E=V_{min}$ (l\'{i}nea punteada de la figura \ref{fig:pot}). Ahora suponga que hay una incremento de la energ\'{i}a total en una cantidad $\Delta E$, este incremento generar\'{a} un aumento en la energ\'{i}a potencial, l\'{i}nea continua.  Si el sistema se desv\'{i}a de la posici\'{o}n de equilibrio, por conservaci\'{o}n de la energ\'{i}a y como $E=T+V$, la energ\'{i}a cin\'{e}tica disminuye. En un equilibrio inestable la energ\'{i}a cin\'{e}tica aumenta, alejando al sistema del equilibrio. \cite{Goldstein2001ClassicalMechanics} \\


\begin{figure}
\centering%
% GNUPLOT: LaTeX picture
\setlength{\unitlength}{0.240900pt}
\ifx\plotpoint\undefined\newsavebox{\plotpoint}\fi
\begin{picture}(1500,900)(0,0)
\sbox{\plotpoint}{\rule[-0.200pt]{0.400pt}{0.400pt}}%
\put(151.0,131.0){\rule[-0.200pt]{4.818pt}{0.400pt}}
\put(131,131){\makebox(0,0)[r]{$0$}}
\put(1419.0,131.0){\rule[-0.200pt]{4.818pt}{0.400pt}}
\put(151.0,252.0){\rule[-0.200pt]{4.818pt}{0.400pt}}
\put(131,252){\makebox(0,0)[r]{$0.2$}}
\put(1419.0,252.0){\rule[-0.200pt]{4.818pt}{0.400pt}}
\put(151.0,374.0){\rule[-0.200pt]{4.818pt}{0.400pt}}
\put(131,374){\makebox(0,0)[r]{$0.4$}}
\put(1419.0,374.0){\rule[-0.200pt]{4.818pt}{0.400pt}}
\put(151.0,495.0){\rule[-0.200pt]{4.818pt}{0.400pt}}
\put(131,495){\makebox(0,0)[r]{$0.6$}}
\put(1419.0,495.0){\rule[-0.200pt]{4.818pt}{0.400pt}}
\put(151.0,616.0){\rule[-0.200pt]{4.818pt}{0.400pt}}
\put(131,616){\makebox(0,0)[r]{$0.8$}}
\put(1419.0,616.0){\rule[-0.200pt]{4.818pt}{0.400pt}}
\put(151.0,738.0){\rule[-0.200pt]{4.818pt}{0.400pt}}
\put(131,738){\makebox(0,0)[r]{$1$}}
\put(1419.0,738.0){\rule[-0.200pt]{4.818pt}{0.400pt}}
\put(151.0,859.0){\rule[-0.200pt]{4.818pt}{0.400pt}}
\put(131,859){\makebox(0,0)[r]{$1.2$}}
\put(1419.0,859.0){\rule[-0.200pt]{4.818pt}{0.400pt}}
\put(151.0,131.0){\rule[-0.200pt]{0.400pt}{4.818pt}}
\put(151,90){\makebox(0,0){$0$}}
\put(151.0,839.0){\rule[-0.200pt]{0.400pt}{4.818pt}}
\put(473.0,131.0){\rule[-0.200pt]{0.400pt}{4.818pt}}
\put(473,90){\makebox(0,0){$0.5$}}
\put(473.0,839.0){\rule[-0.200pt]{0.400pt}{4.818pt}}
\put(795.0,131.0){\rule[-0.200pt]{0.400pt}{4.818pt}}
\put(795,90){\makebox(0,0){$1$}}
\put(795.0,839.0){\rule[-0.200pt]{0.400pt}{4.818pt}}
\put(1117.0,131.0){\rule[-0.200pt]{0.400pt}{4.818pt}}
\put(1117,90){\makebox(0,0){$1.5$}}
\put(1117.0,839.0){\rule[-0.200pt]{0.400pt}{4.818pt}}
\put(1439.0,131.0){\rule[-0.200pt]{0.400pt}{4.818pt}}
\put(1439,90){\makebox(0,0){$2$}}
\put(1439.0,839.0){\rule[-0.200pt]{0.400pt}{4.818pt}}
\put(151.0,131.0){\rule[-0.200pt]{0.400pt}{175.375pt}}
\put(151.0,131.0){\rule[-0.200pt]{310.279pt}{0.400pt}}
\put(1439.0,131.0){\rule[-0.200pt]{0.400pt}{175.375pt}}
\put(151.0,859.0){\rule[-0.200pt]{310.279pt}{0.400pt}}
\put(30,495){\makebox(0,0){$E(q)$}}
\put(795,29){\makebox(0,0){$q$}}
\put(1279,819){\makebox(0,0)[r]{x**2-2*x+1}}
\put(1299.0,819.0){\rule[-0.200pt]{24.090pt}{0.400pt}}
\put(151,738){\usebox{\plotpoint}}
\multiput(151.58,734.39)(0.493,-0.972){23}{\rule{0.119pt}{0.869pt}}
\multiput(150.17,736.20)(13.000,-23.196){2}{\rule{0.400pt}{0.435pt}}
\multiput(164.58,709.65)(0.493,-0.893){23}{\rule{0.119pt}{0.808pt}}
\multiput(163.17,711.32)(13.000,-21.324){2}{\rule{0.400pt}{0.404pt}}
\multiput(177.58,686.52)(0.493,-0.933){23}{\rule{0.119pt}{0.838pt}}
\multiput(176.17,688.26)(13.000,-22.260){2}{\rule{0.400pt}{0.419pt}}
\multiput(190.58,662.77)(0.493,-0.853){23}{\rule{0.119pt}{0.777pt}}
\multiput(189.17,664.39)(13.000,-20.387){2}{\rule{0.400pt}{0.388pt}}
\multiput(203.58,640.65)(0.493,-0.893){23}{\rule{0.119pt}{0.808pt}}
\multiput(202.17,642.32)(13.000,-21.324){2}{\rule{0.400pt}{0.404pt}}
\multiput(216.58,617.90)(0.493,-0.814){23}{\rule{0.119pt}{0.746pt}}
\multiput(215.17,619.45)(13.000,-19.451){2}{\rule{0.400pt}{0.373pt}}
\multiput(229.58,596.77)(0.493,-0.853){23}{\rule{0.119pt}{0.777pt}}
\multiput(228.17,598.39)(13.000,-20.387){2}{\rule{0.400pt}{0.388pt}}
\multiput(242.58,574.90)(0.493,-0.814){23}{\rule{0.119pt}{0.746pt}}
\multiput(241.17,576.45)(13.000,-19.451){2}{\rule{0.400pt}{0.373pt}}
\multiput(255.58,554.03)(0.493,-0.774){23}{\rule{0.119pt}{0.715pt}}
\multiput(254.17,555.52)(13.000,-18.515){2}{\rule{0.400pt}{0.358pt}}
\multiput(268.58,534.03)(0.493,-0.774){23}{\rule{0.119pt}{0.715pt}}
\multiput(267.17,535.52)(13.000,-18.515){2}{\rule{0.400pt}{0.358pt}}
\multiput(281.58,514.16)(0.493,-0.734){23}{\rule{0.119pt}{0.685pt}}
\multiput(280.17,515.58)(13.000,-17.579){2}{\rule{0.400pt}{0.342pt}}
\multiput(294.58,495.16)(0.493,-0.734){23}{\rule{0.119pt}{0.685pt}}
\multiput(293.17,496.58)(13.000,-17.579){2}{\rule{0.400pt}{0.342pt}}
\multiput(307.58,476.29)(0.493,-0.695){23}{\rule{0.119pt}{0.654pt}}
\multiput(306.17,477.64)(13.000,-16.643){2}{\rule{0.400pt}{0.327pt}}
\multiput(320.58,458.29)(0.493,-0.695){23}{\rule{0.119pt}{0.654pt}}
\multiput(319.17,459.64)(13.000,-16.643){2}{\rule{0.400pt}{0.327pt}}
\multiput(333.58,440.41)(0.493,-0.655){23}{\rule{0.119pt}{0.623pt}}
\multiput(332.17,441.71)(13.000,-15.707){2}{\rule{0.400pt}{0.312pt}}
\multiput(346.58,423.41)(0.493,-0.655){23}{\rule{0.119pt}{0.623pt}}
\multiput(345.17,424.71)(13.000,-15.707){2}{\rule{0.400pt}{0.312pt}}
\multiput(359.58,406.54)(0.493,-0.616){23}{\rule{0.119pt}{0.592pt}}
\multiput(358.17,407.77)(13.000,-14.771){2}{\rule{0.400pt}{0.296pt}}
\multiput(372.58,390.54)(0.493,-0.616){23}{\rule{0.119pt}{0.592pt}}
\multiput(371.17,391.77)(13.000,-14.771){2}{\rule{0.400pt}{0.296pt}}
\multiput(385.58,374.54)(0.493,-0.616){23}{\rule{0.119pt}{0.592pt}}
\multiput(384.17,375.77)(13.000,-14.771){2}{\rule{0.400pt}{0.296pt}}
\multiput(398.58,358.67)(0.493,-0.576){23}{\rule{0.119pt}{0.562pt}}
\multiput(397.17,359.83)(13.000,-13.834){2}{\rule{0.400pt}{0.281pt}}
\multiput(411.58,343.80)(0.493,-0.536){23}{\rule{0.119pt}{0.531pt}}
\multiput(410.17,344.90)(13.000,-12.898){2}{\rule{0.400pt}{0.265pt}}
\multiput(424.58,329.80)(0.493,-0.536){23}{\rule{0.119pt}{0.531pt}}
\multiput(423.17,330.90)(13.000,-12.898){2}{\rule{0.400pt}{0.265pt}}
\multiput(437.00,316.92)(0.497,-0.493){23}{\rule{0.500pt}{0.119pt}}
\multiput(437.00,317.17)(11.962,-13.000){2}{\rule{0.250pt}{0.400pt}}
\multiput(450.00,303.92)(0.497,-0.493){23}{\rule{0.500pt}{0.119pt}}
\multiput(450.00,304.17)(11.962,-13.000){2}{\rule{0.250pt}{0.400pt}}
\multiput(463.00,290.92)(0.539,-0.492){21}{\rule{0.533pt}{0.119pt}}
\multiput(463.00,291.17)(11.893,-12.000){2}{\rule{0.267pt}{0.400pt}}
\multiput(476.00,278.92)(0.539,-0.492){21}{\rule{0.533pt}{0.119pt}}
\multiput(476.00,279.17)(11.893,-12.000){2}{\rule{0.267pt}{0.400pt}}
\multiput(489.00,266.92)(0.539,-0.492){21}{\rule{0.533pt}{0.119pt}}
\multiput(489.00,267.17)(11.893,-12.000){2}{\rule{0.267pt}{0.400pt}}
\multiput(502.00,254.92)(0.590,-0.492){19}{\rule{0.573pt}{0.118pt}}
\multiput(502.00,255.17)(11.811,-11.000){2}{\rule{0.286pt}{0.400pt}}
\multiput(515.00,243.92)(0.652,-0.491){17}{\rule{0.620pt}{0.118pt}}
\multiput(515.00,244.17)(11.713,-10.000){2}{\rule{0.310pt}{0.400pt}}
\multiput(528.00,233.92)(0.652,-0.491){17}{\rule{0.620pt}{0.118pt}}
\multiput(528.00,234.17)(11.713,-10.000){2}{\rule{0.310pt}{0.400pt}}
\multiput(541.00,223.93)(0.728,-0.489){15}{\rule{0.678pt}{0.118pt}}
\multiput(541.00,224.17)(11.593,-9.000){2}{\rule{0.339pt}{0.400pt}}
\multiput(554.00,214.93)(0.728,-0.489){15}{\rule{0.678pt}{0.118pt}}
\multiput(554.00,215.17)(11.593,-9.000){2}{\rule{0.339pt}{0.400pt}}
\multiput(567.00,205.93)(0.728,-0.489){15}{\rule{0.678pt}{0.118pt}}
\multiput(567.00,206.17)(11.593,-9.000){2}{\rule{0.339pt}{0.400pt}}
\multiput(580.00,196.93)(0.824,-0.488){13}{\rule{0.750pt}{0.117pt}}
\multiput(580.00,197.17)(11.443,-8.000){2}{\rule{0.375pt}{0.400pt}}
\multiput(593.00,188.93)(0.950,-0.485){11}{\rule{0.843pt}{0.117pt}}
\multiput(593.00,189.17)(11.251,-7.000){2}{\rule{0.421pt}{0.400pt}}
\multiput(606.00,181.93)(0.950,-0.485){11}{\rule{0.843pt}{0.117pt}}
\multiput(606.00,182.17)(11.251,-7.000){2}{\rule{0.421pt}{0.400pt}}
\multiput(619.00,174.93)(1.123,-0.482){9}{\rule{0.967pt}{0.116pt}}
\multiput(619.00,175.17)(10.994,-6.000){2}{\rule{0.483pt}{0.400pt}}
\multiput(632.00,168.93)(1.123,-0.482){9}{\rule{0.967pt}{0.116pt}}
\multiput(632.00,169.17)(10.994,-6.000){2}{\rule{0.483pt}{0.400pt}}
\multiput(645.00,162.93)(1.123,-0.482){9}{\rule{0.967pt}{0.116pt}}
\multiput(645.00,163.17)(10.994,-6.000){2}{\rule{0.483pt}{0.400pt}}
\multiput(658.00,156.93)(1.378,-0.477){7}{\rule{1.140pt}{0.115pt}}
\multiput(658.00,157.17)(10.634,-5.000){2}{\rule{0.570pt}{0.400pt}}
\multiput(671.00,151.94)(1.797,-0.468){5}{\rule{1.400pt}{0.113pt}}
\multiput(671.00,152.17)(10.094,-4.000){2}{\rule{0.700pt}{0.400pt}}
\multiput(684.00,147.94)(1.797,-0.468){5}{\rule{1.400pt}{0.113pt}}
\multiput(684.00,148.17)(10.094,-4.000){2}{\rule{0.700pt}{0.400pt}}
\multiput(697.00,143.94)(1.797,-0.468){5}{\rule{1.400pt}{0.113pt}}
\multiput(697.00,144.17)(10.094,-4.000){2}{\rule{0.700pt}{0.400pt}}
\multiput(710.00,139.95)(2.695,-0.447){3}{\rule{1.833pt}{0.108pt}}
\multiput(710.00,140.17)(9.195,-3.000){2}{\rule{0.917pt}{0.400pt}}
\put(723,136.17){\rule{2.700pt}{0.400pt}}
\multiput(723.00,137.17)(7.396,-2.000){2}{\rule{1.350pt}{0.400pt}}
\put(736,134.17){\rule{2.700pt}{0.400pt}}
\multiput(736.00,135.17)(7.396,-2.000){2}{\rule{1.350pt}{0.400pt}}
\put(749,132.67){\rule{3.132pt}{0.400pt}}
\multiput(749.00,133.17)(6.500,-1.000){2}{\rule{1.566pt}{0.400pt}}
\put(762,131.67){\rule{3.132pt}{0.400pt}}
\multiput(762.00,132.17)(6.500,-1.000){2}{\rule{1.566pt}{0.400pt}}
\put(775,130.67){\rule{3.132pt}{0.400pt}}
\multiput(775.00,131.17)(6.500,-1.000){2}{\rule{1.566pt}{0.400pt}}
\put(802,130.67){\rule{3.132pt}{0.400pt}}
\multiput(802.00,130.17)(6.500,1.000){2}{\rule{1.566pt}{0.400pt}}
\put(815,131.67){\rule{3.132pt}{0.400pt}}
\multiput(815.00,131.17)(6.500,1.000){2}{\rule{1.566pt}{0.400pt}}
\put(828,132.67){\rule{3.132pt}{0.400pt}}
\multiput(828.00,132.17)(6.500,1.000){2}{\rule{1.566pt}{0.400pt}}
\put(841,134.17){\rule{2.700pt}{0.400pt}}
\multiput(841.00,133.17)(7.396,2.000){2}{\rule{1.350pt}{0.400pt}}
\put(854,136.17){\rule{2.700pt}{0.400pt}}
\multiput(854.00,135.17)(7.396,2.000){2}{\rule{1.350pt}{0.400pt}}
\multiput(867.00,138.61)(2.695,0.447){3}{\rule{1.833pt}{0.108pt}}
\multiput(867.00,137.17)(9.195,3.000){2}{\rule{0.917pt}{0.400pt}}
\multiput(880.00,141.60)(1.797,0.468){5}{\rule{1.400pt}{0.113pt}}
\multiput(880.00,140.17)(10.094,4.000){2}{\rule{0.700pt}{0.400pt}}
\multiput(893.00,145.60)(1.797,0.468){5}{\rule{1.400pt}{0.113pt}}
\multiput(893.00,144.17)(10.094,4.000){2}{\rule{0.700pt}{0.400pt}}
\multiput(906.00,149.60)(1.797,0.468){5}{\rule{1.400pt}{0.113pt}}
\multiput(906.00,148.17)(10.094,4.000){2}{\rule{0.700pt}{0.400pt}}
\multiput(919.00,153.59)(1.378,0.477){7}{\rule{1.140pt}{0.115pt}}
\multiput(919.00,152.17)(10.634,5.000){2}{\rule{0.570pt}{0.400pt}}
\multiput(932.00,158.59)(1.123,0.482){9}{\rule{0.967pt}{0.116pt}}
\multiput(932.00,157.17)(10.994,6.000){2}{\rule{0.483pt}{0.400pt}}
\multiput(945.00,164.59)(1.123,0.482){9}{\rule{0.967pt}{0.116pt}}
\multiput(945.00,163.17)(10.994,6.000){2}{\rule{0.483pt}{0.400pt}}
\multiput(958.00,170.59)(1.123,0.482){9}{\rule{0.967pt}{0.116pt}}
\multiput(958.00,169.17)(10.994,6.000){2}{\rule{0.483pt}{0.400pt}}
\multiput(971.00,176.59)(0.950,0.485){11}{\rule{0.843pt}{0.117pt}}
\multiput(971.00,175.17)(11.251,7.000){2}{\rule{0.421pt}{0.400pt}}
\multiput(984.00,183.59)(0.950,0.485){11}{\rule{0.843pt}{0.117pt}}
\multiput(984.00,182.17)(11.251,7.000){2}{\rule{0.421pt}{0.400pt}}
\multiput(997.00,190.59)(0.824,0.488){13}{\rule{0.750pt}{0.117pt}}
\multiput(997.00,189.17)(11.443,8.000){2}{\rule{0.375pt}{0.400pt}}
\multiput(1010.00,198.59)(0.728,0.489){15}{\rule{0.678pt}{0.118pt}}
\multiput(1010.00,197.17)(11.593,9.000){2}{\rule{0.339pt}{0.400pt}}
\multiput(1023.00,207.59)(0.728,0.489){15}{\rule{0.678pt}{0.118pt}}
\multiput(1023.00,206.17)(11.593,9.000){2}{\rule{0.339pt}{0.400pt}}
\multiput(1036.00,216.59)(0.728,0.489){15}{\rule{0.678pt}{0.118pt}}
\multiput(1036.00,215.17)(11.593,9.000){2}{\rule{0.339pt}{0.400pt}}
\multiput(1049.00,225.58)(0.652,0.491){17}{\rule{0.620pt}{0.118pt}}
\multiput(1049.00,224.17)(11.713,10.000){2}{\rule{0.310pt}{0.400pt}}
\multiput(1062.00,235.58)(0.652,0.491){17}{\rule{0.620pt}{0.118pt}}
\multiput(1062.00,234.17)(11.713,10.000){2}{\rule{0.310pt}{0.400pt}}
\multiput(1075.00,245.58)(0.590,0.492){19}{\rule{0.573pt}{0.118pt}}
\multiput(1075.00,244.17)(11.811,11.000){2}{\rule{0.286pt}{0.400pt}}
\multiput(1088.00,256.58)(0.539,0.492){21}{\rule{0.533pt}{0.119pt}}
\multiput(1088.00,255.17)(11.893,12.000){2}{\rule{0.267pt}{0.400pt}}
\multiput(1101.00,268.58)(0.539,0.492){21}{\rule{0.533pt}{0.119pt}}
\multiput(1101.00,267.17)(11.893,12.000){2}{\rule{0.267pt}{0.400pt}}
\multiput(1114.00,280.58)(0.539,0.492){21}{\rule{0.533pt}{0.119pt}}
\multiput(1114.00,279.17)(11.893,12.000){2}{\rule{0.267pt}{0.400pt}}
\multiput(1127.00,292.58)(0.497,0.493){23}{\rule{0.500pt}{0.119pt}}
\multiput(1127.00,291.17)(11.962,13.000){2}{\rule{0.250pt}{0.400pt}}
\multiput(1140.00,305.58)(0.497,0.493){23}{\rule{0.500pt}{0.119pt}}
\multiput(1140.00,304.17)(11.962,13.000){2}{\rule{0.250pt}{0.400pt}}
\multiput(1153.58,318.00)(0.493,0.536){23}{\rule{0.119pt}{0.531pt}}
\multiput(1152.17,318.00)(13.000,12.898){2}{\rule{0.400pt}{0.265pt}}
\multiput(1166.58,332.00)(0.493,0.536){23}{\rule{0.119pt}{0.531pt}}
\multiput(1165.17,332.00)(13.000,12.898){2}{\rule{0.400pt}{0.265pt}}
\multiput(1179.58,346.00)(0.493,0.576){23}{\rule{0.119pt}{0.562pt}}
\multiput(1178.17,346.00)(13.000,13.834){2}{\rule{0.400pt}{0.281pt}}
\multiput(1192.58,361.00)(0.493,0.616){23}{\rule{0.119pt}{0.592pt}}
\multiput(1191.17,361.00)(13.000,14.771){2}{\rule{0.400pt}{0.296pt}}
\multiput(1205.58,377.00)(0.493,0.616){23}{\rule{0.119pt}{0.592pt}}
\multiput(1204.17,377.00)(13.000,14.771){2}{\rule{0.400pt}{0.296pt}}
\multiput(1218.58,393.00)(0.493,0.616){23}{\rule{0.119pt}{0.592pt}}
\multiput(1217.17,393.00)(13.000,14.771){2}{\rule{0.400pt}{0.296pt}}
\multiput(1231.58,409.00)(0.493,0.655){23}{\rule{0.119pt}{0.623pt}}
\multiput(1230.17,409.00)(13.000,15.707){2}{\rule{0.400pt}{0.312pt}}
\multiput(1244.58,426.00)(0.493,0.655){23}{\rule{0.119pt}{0.623pt}}
\multiput(1243.17,426.00)(13.000,15.707){2}{\rule{0.400pt}{0.312pt}}
\multiput(1257.58,443.00)(0.493,0.695){23}{\rule{0.119pt}{0.654pt}}
\multiput(1256.17,443.00)(13.000,16.643){2}{\rule{0.400pt}{0.327pt}}
\multiput(1270.58,461.00)(0.493,0.695){23}{\rule{0.119pt}{0.654pt}}
\multiput(1269.17,461.00)(13.000,16.643){2}{\rule{0.400pt}{0.327pt}}
\multiput(1283.58,479.00)(0.493,0.734){23}{\rule{0.119pt}{0.685pt}}
\multiput(1282.17,479.00)(13.000,17.579){2}{\rule{0.400pt}{0.342pt}}
\multiput(1296.58,498.00)(0.493,0.734){23}{\rule{0.119pt}{0.685pt}}
\multiput(1295.17,498.00)(13.000,17.579){2}{\rule{0.400pt}{0.342pt}}
\multiput(1309.58,517.00)(0.493,0.774){23}{\rule{0.119pt}{0.715pt}}
\multiput(1308.17,517.00)(13.000,18.515){2}{\rule{0.400pt}{0.358pt}}
\multiput(1322.58,537.00)(0.493,0.774){23}{\rule{0.119pt}{0.715pt}}
\multiput(1321.17,537.00)(13.000,18.515){2}{\rule{0.400pt}{0.358pt}}
\multiput(1335.58,557.00)(0.493,0.814){23}{\rule{0.119pt}{0.746pt}}
\multiput(1334.17,557.00)(13.000,19.451){2}{\rule{0.400pt}{0.373pt}}
\multiput(1348.58,578.00)(0.493,0.853){23}{\rule{0.119pt}{0.777pt}}
\multiput(1347.17,578.00)(13.000,20.387){2}{\rule{0.400pt}{0.388pt}}
\multiput(1361.58,600.00)(0.493,0.814){23}{\rule{0.119pt}{0.746pt}}
\multiput(1360.17,600.00)(13.000,19.451){2}{\rule{0.400pt}{0.373pt}}
\multiput(1374.58,621.00)(0.493,0.893){23}{\rule{0.119pt}{0.808pt}}
\multiput(1373.17,621.00)(13.000,21.324){2}{\rule{0.400pt}{0.404pt}}
\multiput(1387.58,644.00)(0.493,0.853){23}{\rule{0.119pt}{0.777pt}}
\multiput(1386.17,644.00)(13.000,20.387){2}{\rule{0.400pt}{0.388pt}}
\multiput(1400.58,666.00)(0.493,0.933){23}{\rule{0.119pt}{0.838pt}}
\multiput(1399.17,666.00)(13.000,22.260){2}{\rule{0.400pt}{0.419pt}}
\multiput(1413.58,690.00)(0.493,0.893){23}{\rule{0.119pt}{0.808pt}}
\multiput(1412.17,690.00)(13.000,21.324){2}{\rule{0.400pt}{0.404pt}}
\multiput(1426.58,713.00)(0.493,0.972){23}{\rule{0.119pt}{0.869pt}}
\multiput(1425.17,713.00)(13.000,23.196){2}{\rule{0.400pt}{0.435pt}}
\put(788.0,131.0){\rule[-0.200pt]{3.373pt}{0.400pt}}
\put(151.0,131.0){\rule[-0.200pt]{0.400pt}{175.375pt}}
\put(151.0,131.0){\rule[-0.200pt]{310.279pt}{0.400pt}}
\put(1439.0,131.0){\rule[-0.200pt]{0.400pt}{175.375pt}}
\put(151.0,859.0){\rule[-0.200pt]{310.279pt}{0.400pt}}
\end{picture}
%
\put(-870,330){$E$}
\put(-1000,480){$E+\Delta E$}
\put(-400,390){$T$}
\put(-405,345){\vector(0,1){130}}
\put(-280,410){$T$}
\put(-300,375){\vector(0,1){100}}
\caption{Potencial en funci\'{o}n de la posici\'{o}n y energ\'{i}a total $E$} \label{fig:pot}
\end{figure}

Considerando la condici\'{o}n de equilibrio \eqref{eq:2} y despreciando desplazamientos de orden superior se tiene que:
\begin{equation}
V(\mathbf{q})=\sum_{i,j=1}^{n}\frac{\partial^2 V }{\partial q_i\partial q_j}q_i q_j
\end{equation}\label{eq:3}

Donde se identifican las constantes el\'{a}sticas como:
\begin{equation}
H_{ij}=\frac{\partial^2 V }{\partial q_i\partial q_j}
\end{equation}\label{eq:4}

En t\'{e}rminos matriciales el potencial se puede escribir como:
\begin{equation}
V(\mathbf{q})=\mathbf{q}^t\mathbf{H}\mathbf{q}
\end{equation}\label{eq:5}
En la ecuaci\'{o}n \eqref{eq:5} $\mathbf{q}$ es el vector columna formado por los desplazamientos de las posiciones de equilibrio para cada constituyente, i. e., $\mathbf{q}^t=(q_1,q_2,...q_n)$. Por otro lado la matriz de constantes el\'{a}sticas $\mathbf{H}$, es una matriz sim\'{e}trica debido a que la fuerza generalizada se considera conservativa, lo cual permite intercambiar el orden de las derivadas parciales.\\

Para las peque\~{n}as oscilaciones, no existen ligaduras dependientes expl\'{i}citamente de el tiempo (holon\'{o}micas) luego, la eneg\'{i}a cin\'{e}tica de los constituyentes s\'{o}lo depender\'{a} de los cuadrados de las velocidades generalizadas:
\begin{equation}
T=\frac{1}{2}\mathbf{q}^t\mathbf{M}\mathbf{q}
\end{equation}\label{eq:6}
Donde $\mathbf{M}$ es la masa generalizada, la cual se expresa en t\'{e}rminos de los factores de escala entre sistemas coordenados:
\begin{equation}
M_{jk}=\sum_{i=1}^{N} m_{i}\frac{\partial \mathbf{r_{i}} }{\partial q_j}\cdot\frac{\partial \mathbf{r_{i}} }{\partial q_k}
\end{equation}\label{eq:7}
La forma en que se escribir\'{a}n los elementos de la masa del sistema depende de la transformaci\'{o}n aplicada. Por ejemplo, en una dimensi\'{o}n las componentes de $\mathbf{q}$ se definen como:
\begin{equation}
q_i=x_i-x_{i0}
\end{equation}\label{eq:8}
En \eqref{eq:8} $x_i$ es la posici\'{o}n variable de la part\'{i}cula $i$ con respecto a un sistema fijo y $x_{i0}$ su posici\'{o}n de equilibrio. Para este caso particular \eqref{eq:7} se convierte en:
\begin{eqnarray}
M_{jk}&=&\sum_{i=1}^{N} m_{i} \delta_{ij}\delta_{ik}\nonumber \\
M_{jk}&=&m_{j} \delta_{jk}
\end{eqnarray}\label{eq:9}
La equaci\'{o}n \eqref{eq:9} dice que la matriz $\mathbf{M}$ es una matriz diagonal cuyas componentes son las masas del sistema.\\
En tres dimensiones, como es nuestro caso real, la transformaci\'{o}n depender\'{a} del modelo escogido. Exceptuando el modelo de redes gaussianas (Gaussian Network Model que por sus siglas en ingl\'{e}s es GNM) la transformaci\'{o}n de coordenadas va de $\mathbf{r}_{i}$ posiciones con $i=1,2,...,N$ ($3N$ coordenadas) a $q_j$ coordenadas $j=1,2,..,3N$:
\begin{equation*}
\mathbf{r}_{i}\longrightarrow q_{j}
\end{equation*}
Con
\begin{equation*}
i=1,2,...,N\mbox{  }j=1,2,..,3N
\end{equation*}
Por cada componente en cartesianas:
\begin{eqnarray}
\begin{array}{cccccc}
q_1=x_1-x_{10}&q_4=x_2-x_{20}&\cdots &q_{3i-2}=x_i-x_{i0}&\cdots &q_{3N-2}=x_N-x_{N0} \\
q_2=y_1-y_{10}&q_5=y_2-y_{20}&\cdots &q_{3i-1}=y_i-y_{i0}&\cdots &q_{3N-1}=y_N-y_{N0}\\
q_3=z_1-z_{10}&q_6=z_2-z_{20}&\cdots &q_{3i}=z_i-z_{i0}&\cdots &q_{3N}=z_N-z_{N0}\\
\end{array}
\end{eqnarray}\label{eq:10}
Para esta transformaci\'{o}n, \eqref{eq:7} se convierte en:
\begin{eqnarray}
M_{jk}&=&\sum_{i=1}^{N} m_{i}\left( \delta_{i,3j-2}\delta_{jk}+\delta_{i,3j-1}\delta_{jk}+  \delta_{i,3j}\delta_{jk}\right)\nonumber \\
M_{jk}&=&m_{3j-2}\delta_{jk}+m_{3j-1}\delta_{jk}+m_{3j}\delta_{jk} \nonumber \\
M_{jk}&=&\left( m_{3j-2}+m_{3j-1}+m_{3j} \right) \delta_{jk}
\end{eqnarray}\label{eq:11}
En \eqref{eq:11} debe resaltarse que para $j=k=3N$, el elemento de matriz $M_{3N,3N}$ requiere las masas $m_{3(3N)-2}=m_{9N-2}$, $m_{3(3N)-1}=m_{9N-1}$ y $m_{3(3N)}=m_{9N}$, sin embargo ! no hay $9N$ masas!, el n\'{u}mero de masas es el mismo n\'{u}mero de nodos: $N$, entonces, para poder calcular la matriz $\mathbf{M}$ es necesario definir lo siguiente:
\begin{equation}
m_{N+1},m_{N+2},...,m_{3N}=0
\end{equation}\label{eq:12}
Como a partir de $N+1$ las masas son nulas, la matriz de masa (que es diagonal) tiene elementos nulos si  
\begin{eqnarray*}
3j-2=N+1\\
j=\frac{N+3}{3} \\
\end{eqnarray*}
Como no siempre $N$ es m\'{u}ltiplo de 3, se escoge el entero menor que este m\'{a}s cerca al valor:
\begin{equation}
j=\left \lfloor\frac{N+3}{3}\right \rfloor
\end{equation}\label{eq:13}
\paragraph{Ecuaci\'{o}n de Movimiento}
El sistema satisface la ecuaci\'{o}n de un oscilador arm\'{o}nico, tal como es descrito en \cite{Goldstein2001ClassicalMechanics}:
\begin{equation}
\mathbf{q}\mathbf{M}+\mathbf{q}\mathbf{H}=\mathbf{0}
\end{equation}\label{eq:14}
\subsubsection{Ensamble Estad\'{i}stico}
\subsubsection{Modelo de Redes Gaussianas (GNM)}
\subsubsection{Modelos de Redes Anisotr\'{o}picas (ANM)}
\subsection{An\'{a}lisis de Modos Normales Est\'{a}ndar (NMA)}
Describe un sistema oscilatorio en el que todos los constituyentes del sistema oscilan sinusoidalmente y con la misma frecuencia.
\subsection{An\'{a}lisis por Componentes Principales (PCA)}
\section{Movimientos Locales}
Hace referencia a las simulaciones en las que se incluyen todos los \'{a}tomos junto con las interacciones presentes, es decir, en las que se analizan los \textit{cambios locales}. Estas se pueden simular a un orden de magnitud de los nanosegundos en una m\'{a}quina usual, al respecto ver \cite{Gur2013GlobalPredictions.}.

Como caso particular se pueden tomar los potenciales usados en \cite{Amber2016AmberManual}, que siguen el modelo de Amber. El modelo de Amber tiene en cuenta las contribuciones debidas a:
 \begin{itemize}
\item Interacciones intermoleculares: Son las producidas por los enlaces covalentes entre grupos de \'{a}tomos, las de valencia y las torsiones.
\item Interacciones entre pares: Lennard Jones, electrost\'{a}tico.
\end{itemize}

\subsubsection{An\'{a}lisis por Componentes Principales}

\begin{itemize}
\item 
\end{itemize}
