\chapter*{Introducci\'{o}n}
\addcontentsline{toc}{chapter}{\numberline{}Introducci\'{o}n}
Es importante estudiar los movimientos de una biomol\'{e}cula ya que como es se\~{n}alado en \cite{Lezon2009} y en  \cite{Rader2006}, la din\'{a}mica de la mol\'{e}cula vincula la estructura con la funci\'{o}n de la biomol\'{e}cula. La funci\'{o}n es el papel que desempe\~{n}a la biomol\'{e}cula y que est\'{a} intimamente relacionado con las interacciones de la biomol\'{e}cula a un ligando. La estructura de una biomol\'{e}cula tiene 4 niveles de organizaci\'{o}n, denominadas \textit{estructuras primaria, secundaria, terciaria y cuaternaria} y dice la forma en la que se ordenan los mon\'{o}meros que la constituyen. La estructura se determina por diversos m\'{e}todos como la cristalograf\'{i}a de rayos x y la resonancia megn\'{e}tica nuclear.\\


El paradigma de la estructura con la funci\'{o}n ocurre ya que un cambio en la secuencia de un mon\'{o}mero puede causar un reordenamiento en la geometr\'{i}a global, causando la p\'{e}rdida o no de su funci\'{o}n biol\'{o}gica, \cite{Dykeman2010NormalPhysics}. Por otro lado, la estructura no es suficiente para determinar la funci\'{o}n, esto debido a que las estructuras no act\'{u}an biol\'{o}gicamente de forma est\'{a}tica sino m\'{o}vil o din\'{a}micamente. Tal es el la importancia de la din\'{a}mica, que \'{e}sta funciona como un v\'{i}nculo en la estructura con la funci\'{o}n, \cite{Bahar2005Coarse-grainedBiology}.\\


Muchos de los movimientos relevantes en una biomol\'{e}cula ocurren en sus dominios o una peque\~{n}a secci\'{o}n de ella. En esta secci\'{o}n se presentan movimientos que 
permiten en las prote\'{i}nas, por ejemplo, transportar, catalizar una reacci\'{o}n o envolver otra mol\'{e}cula. El estudio de estos movimientos se ha realizado desde comienzos de los a\~{n}os 80, con las simulaciones por din\'{a}mica molecular usadas por primera vez para estudiar el inhibidor de la tripsina pancre\'{a}tica bovina,  \cite{Bahar2005Coarse-grainedBiology}.\\

Las simulaciones por din\'{a}mica molecular son una herramienta \'{u}til cuando las escalas de tiempo son pequ\~{n}as, es decir, entre $10^{-15}$ y  $10^{-6}$ segundos y la escala va desde la at\'{o}mica hasta unos cientos de Angstroms. Incluso, en los l\'{i}mites de estos rangos una smulaci\'{o}n por din\'{a}mica molecular puede volverse tediosa, debido al costo y tiempo computacional.\\

Una alternativa a las simulaciones por din\'{a}mica molecular son los \textit{modelos de grano grueso} o en ing\'{e}s \textit{coarse grained models}.  Mientras que en la din\'{a}mica molecular se usan expl\'{i}citamente todos los \'{a}tomos, los modelos de grano grueso reemplazan la representaci\'{o}n atom\'{i}stica por un modelo de m\'{a}s baja resoluci\'{o}n. El objetivo de los modelos de grano grueso es conocer el aspecto global o colectivo de los movimientos de la biomol\'{e}cula.\\