\chapter*{Introducci\'{o}n}\label{ch:1}
\addcontentsline{toc}{chapter}{\numberline{}Introducci\'{o}n}
Los movimientos de una biomol\'{e}cula vinculan la estructura con su funci\'{o}n biol\'{o}gica mediante la din\'{a}mica molecular \cite{Lezon2009}, \cite{Rader2006}. La din\'{a}mica tambi\'{e}n se estudia debido a las ventajas que presentan los experimentos computacionales o \textit{in silico} con respecto a los experimentos convencionales como por ejemplo, el estudio de caracter\'{i}sticas que no pueden ser estudiadas en un laboratorio.\\

La funci\'{o}n biol\'{o}gica es el papel que desempe\~{n}a la mol\'{e}cula en su sistema y que est\'{a} intimamente relacionado con el entorno, mientras que su estructura determina la forma en la que se ordenan los \'{a}tomos y los mon\'{o}meros que la constituyen. Esta estructura tiene 4 niveles de organizaci\'{o}n, denominadas \textit{estructuras primaria, secundaria, terciaria y cuaternaria} \cite{Kuchel}. La estructura se puede determinar por diversos m\'{e}todos como la cristalograf\'{i}a de rayos x, la resonancia megn\'{e}tica nuclear, entre otros.\\

El paradigma de la estructura con la funci\'{o}n biol\'{o}gica ocurre ya que un cambio en la secuencia de un mon\'{o}mero puede causar un reordenamiento en la geometr\'{i}a global de la mol\'{e}cula, causando la p\'{e}rdida o no de su funci\'{o}n biol\'{o}gica \cite{Dykeman2010}. Por otro lado, la estructura no es suficiente para determinar la funci\'{o}n biol\'{o}gica de la mol\'{e}cula, esto debido a que las estructuras no act\'{u}an biol\'{o}gicamente de forma est\'{a}tica sino m\'{o}vil o din\'{a}micamente. Tal es  la importancia de la din\'{a}mica, que \'{e}sta funciona como un v\'{i}nculo de la estructura con la funci\'{o}n \cite{Bahar2005Coarse-grainedBiology}.\\

Muchos de los movimientos relevantes en una biomol\'{e}cula ocurren en sus dominios o en una peque\~{n}a porci\'{o}n de ella. En esta porci\'{o}n pueden presentarse movimientos que permiten en las prote\'{i}nas, transportar, catalizar una reacci\'{o}n o envolver otra mol\'{e}cula, entre otros. El estudio de estos movimientos se ha realizado desde comienzos de los a\~{n}os 80 con las simulaciones por din\'{a}mica molecular usadas por primera vez para estudiar el inhibidor de la tripsina pancre\'{a}tica bovina  \cite{Bahar2005Coarse-grainedBiology}.\\

Las simulaciones por din\'{a}mica molecular son una herramienta \'{u}til cuando las escalas de tiempo son pequ\~{n}as, es decir, entre $10^{-15}$ y  $10^{-6}$ segundos y la escala va desde la at\'{o}mica hasta unos cientos de Angstroms. Incluso, en los l\'{i}mites de estos rangos una simulaci\'{o}n por din\'{a}mica molecular puede volverse tediosa, debido al costo y tiempo computacional.\\

Una alternativa a las simulaciones por din\'{a}mica molecular son los \textit{modelos de grano grueso} o en ingl\'{e}s \textit{``Coarse Grained Models''}.  Mientras que en la din\'{a}mica molecular se usan expl\'{i}citamente todos los \'{a}tomos, los modelos de grano grueso reemplazan la representaci\'{o}n atom\'{i}stica por un modelo de m\'{a}s baja resoluci\'{o}n. El objetivo de los modelos de grano grueso es conocer el aspecto global o colectivo de los movimientos de la biomol\'{e}cula.\\

Otra alternativa a las simulaciones por din\'{a}mica molecular es el An\'{a}lisis por Modos Normales (NMA). Este reemplaza los potenciales at\'{o}micos detallados por una aproximaci\'{o}n arm\'{o}nica alrededor de un punto de equilibrio, es decir, alrededor de una estructura estable y de m\'{i}nima energ\'{i}a. La aproximaci\'{o}n alrededor de este punto requiere que los desplazamientos alrededor del equilibrio sean peque\~{n}os \cite{Lezon2009}.\\

Tambi\'{e}n existen modelos que combinan el an\'{a}lisis de modos normales con los modelos de grano grueso, unos de ellos son los modelos de redes el\'{a}sticas. En estos adem\'{a}s de usarse el potencial arm\'{o}nico, tambi\'{e}n se reemplazan la estructura a resoluci\'{o}n at\'{o}mica de la macromol\'{e}cula por los mon\'{o}meros que la conforman \cite{Lezon2009}.\\

Para usar el m\'{e}todo de modos normales se requiere la estructura de la mol\'{e}cula biol\'{o}gica y los resultados esenciales son sus modos vibracionales. Para esto debe conocerse la estructura tridimensional de la mol\'{e}cula biol\'{o}gica.\\

En este trabajo la mol\'{e}cula biol\'{o}gica de estudio es una prote\'{i}na de membrana tipo simportador, conocida como co-transportador de \ce{Na^{+}}/galactosa vSGLT, la cual est\'{a} presente en la bacteria \textit{Vibrio Parahaemolyticus} y se conoce su estructura cristalogr\'{a}fica \cite{Faham2008}, la cual reposa en la base de datos de prote\'{i}nas Protein Data Bank (PDB). Sin embargo, esto no ocurre con otras prote\'{i}nas que pertenecen a su misma familia (Sodium Symporter Family SSS) y que son relevantes biol\'{o}gicamente debido a que son blanco  para tratar enfermedades cr\'{o}nicas como la diabetes tipo II \cite{Bisha2014}, la diarrea \cite{Hamilton2013}, el hipo e hipertiroidismo \cite{Ferrandino2016}, entre otras.\\

La estructura de prote\'{i}nas de membrana es dif\'{i}cil de obtener ya que estas pierden su estructura y su funci\'{o}n biol\'{o}gica al desprenderse de la membrana celular adem\'{a}s de estar frecuentemente a bajas concentraciones, lo cual se intenta solucionar seleccionando detergentes adecuados para conservar estas propiedades \cite{Editor2010}.\\

El estudio del co-transportador vSGLT ha cobrado importancia ya que sirve como modelo de estudio de otras prote\'{i}nas de su familia mediante la predicci\'{o}n de estructura teniendo en cuenta que la mayor\'{i}a de ellas no la tienen, para luego hacer un an\'{a}lisis de los movimientos que den informaci\'{o}n relevante del sistema.\\

Dado que se tiene la informaci\'{o}n de su estructura tridimensional, en el  presente trabajo se realiza un an\'{a}lisis de modos normales de vSGLT usando modelos de red el\'{a}stica con el fin de determinar los movimientos colectivos del co-transportador y analizar si los cambios en los movimientos globales est\'{a}n relacionados con la funci\'{o}n de vSGLT.\\

Este documento se desarrolla de la siguiente manera: En el cap\'{i}tulo \ref{ch:2} se muestra la teor\'{i}a subyacente al NMA, su clasificaci\'{o}n dentro de los m\'{e}todos computacionales para analizar los movimientos de una mol\'{e}cula biol\'{o}gica y se presentan algunos de los m\'{e}todos pertenecientes al NMA. Luego en el cap\'{i}tulo \ref{ch:3} se presentan los estudios del cotransportador vSGLT junto con los resultados generales de la bioqu\'{i}mica de prote\'{i}nas que permiten entender los resultados obtenidos del co-transportador vSGLT. En el cap\'{i}tulo \ref{ch:4} se muestra el an\'{a}lisis de modos normales realizado al co-transportador vSGLT y su interpretaci\'{o}n. Finalmente, en el cap\'{i}tulo \ref{ch:5} se muestran los resultados relevantes encontrados.\\
