\chapter*{Introducci\'{o}n}
EDITAR
El estudio de los cotransportadores de az\'{u}car, en particular, transportadores de glucosa SGLT dependientes de sodio son esenciales en la producci\'{o}n de metabolismo y la energ\'{i}a celular.Los cotransportadores SGLT son miembros de la familia de portadores de soluto (SLC5) y algunos de estos transportadores de inter\'{e}s tienen una secuencia y estructura similar tridimensional similar. En este caso se examin\'{o} el  co-transportador dependiente  sodio galactosa del Vibrio parahaemolyticus (vSGLT), que media el transporte de galactosa en el citoplasma de las bacterias Vibrio parahaemolyticus. Seg\'{u}n la literatura, la cin\'{e}tica del co-transportador tiene entre 5 y 6 estados o conformaciones, pero en este caso de que la desvinculaci\'{o}n de los sustratos se estudia la conformaci\'{o}n, tambi\'{e}n conocido como modelo de liberaci\'{o}n estado de co-transportador que mira hacia dentro. se realiz\'{o} un estudio computacional para analizar los movimientos globales de un transportador vSGLT, y comparamos nuestros resultados computacionales con los que se encuentran en los anteriores informes experimentales. an\'{a}lisis de modos normales con un modelo el\'{a}stico de red (ENM) fue utilizado para explorar los cambios en los movimientos globales entre vSGLT en la presencia o ausencia de los iones que transportan (Na +, galactosa). ENM se ha demostrado que es un c\'{a}lculo \'{u}til herramienta para predecir la din\'{a}mica de las prote\'{i}nas de membrana en muchas aplicaciones. los modos normales m\'{a}s bajas generadas por la ENM proporcionar informaci\'{o}n valiosa sobre la din\'{a}mica global de las biomol\'{e}culas que son relevantes para su funci\'{o}n.\\
\begin{itemize}
\item \textbf{2 a 4 p\'{a}ginas}
\end{itemize}
