\chapter*{Introducci\'{o}n}
\addcontentsline{toc}{chapter}{\numberline{}Introducci\'{o}n}
Es importante estudiar los movimientos de una biomol\'{e}cula ya que como es se\~{n}alado en \cite{Lezon2009} y en  \cite{Rader2006}, la din\'{a}mica de la mol\'{e}cula vincula la estructura con la funci\'{o}n de la biomol\'{e}cula. La funci\'{o}n es el papel que desempe\~{n}a la biomol\'{e}cula y que est\'{a} intimamente relacionado con las interacciones de la biomol\'{e}cula a un ligando. La estructura de una biomol\'{e}cula tiene 4 niveles de organizaci\'{o}n, denominadas \textit{estructuras primaria, secundaria, terciaria y cuaternaria} y dice la forma en la que se ordenan los mon\'{o}meros que la constituyen. La estructura se determina por diversos m\'{e}todos como la cristalograf\'{i}a de rayos x y la resonancia megn\'{e}tica nuclear.\\


El paradigma de la estructura con la funci\'{o}n ocurre ya que un cambio en la secuencia de un mon\'{o}mero puede causar un reordenamiento en la geometr\'{i}a global, causando la p\'{e}rdida o no de su funci\'{o}n biol\'{o}gica, \cite{Dykeman2010NormalPhysics}. Por otro lado, la estructura no es suficiente para determinar la funci\'{o}n, esto debido a que las estructuras no act\'{u}an biol\'{o}gicamente de forma est\'{a}tica sino m\'{o}vil o din\'{a}micamente. Tal es el la importancia de la din\'{a}mica, que \'{e}sta funciona como un v\'{i}nculo en la estructura con la funci\'{o}n, \cite{Bahar2005Coarse-grainedBiology}.\\

 

