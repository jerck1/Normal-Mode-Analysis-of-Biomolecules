%\newpage
%\setcounter{page}{1}
\begin{center}
\begin{figure}
\centering%
\epsfig{file=HojaTitulo/logo.eps,scale=0.25}%
\end{figure}
\thispagestyle{empty} \vspace*{2.0cm} \textbf{\huge
An\'{a}lisis de Modos Normales \vspace{1cm}
en una Biomol\'{e}cula}\\[5.0cm]
\Large\textbf{John Erick Cabrera Ramirez}\\[6.0cm]
\small Universidad Nacional de Colombia\\
Facultad de Ciencias, Departamento de F\'{i}sica\\
Bogot\'{a}, Colombia\\
2017\\
\end{center}

%\newpage{\pagestyle{empty}\cleardoublepage}

\newpage
\begin{center}
\thispagestyle{empty} \vspace*{-0.5cm} \textbf{\huge An\'{a}lisis de Modos Normales \vspace{0.2cm} en una Biomol\'{e}cula}\\[3.0cm]
\Large\textbf{John Erick Cabrera Ramirez}\\[3.0cm]
\small Trabajo de grado presentado como requisito parcial para optar al t\'{\i}tulo de:\\
\textbf{F\'{i}sico}\\[2.5cm]
Directora:\\
PhD, Yuly Edith S\'{a}nchez Mendoza \\[2.0cm]
%T\'{\i}tulo (Ph.D., Doctor, Qu\'{\i}mico, etc.) y nombre del director(a)\\[2.0cm]
L\'{\i}nea de Investigaci\'{o}n:\\
Biof\'{i}sica Molecular\\
%Nombrar la l\'{\i}nea de investigaci\'{o}n en la que enmarca la tesis  o trabajo de investigaci\'{o}n\\
Grupo de Investigaci\'{o}n:\\
Biof\'{i}sica Molecular\\[2.5cm]
Universidad Nacional de Colombia\\
Facultad de Ciencias, Departamento de F\'{i}sica\\
Bogot\'{a}, Colombia\\
2017\\
\end{center}

\newpage
\thispagestyle{empty} \textbf{}\normalsize
\\\\\\%
\textbf{Dedicatoria}\\[4.0cm]

\begin{flushright}
\begin{minipage}{8cm}
    \noindent
        \small
        A mis padres y a mis formadores\\[1.0cm]\\
		Ya que sin ellos no hubiera sido posible realizar el proyecto de tesis tanto materialmente como en la elaboraci\'{o}n de ideas.\\
\end{minipage}
\end{flushright}

%\newpage{\pagestyle{empty}\cleardoublepage}

%\newpage
%\thispagestyle{empty} \textbf{}\normalsize
%\\\\\\%
%\textbf{\LARGE Agradecimientos}
%\addcontentsline{toc}{chapter}{\numberline{}Agradecimientos}\\\\
%Agradezco a la profesora Yuly Edith S\'{a}nchez Mendoza profesora del departamente de f\'{i}sica por ser %una excelente gu\'{i}a en el trabajo, a \\


\newpage
\textbf{\LARGE Resumen}
\addcontentsline{toc}{chapter}{\numberline{}Resumen}\\\\

El estudio de los cotransportadores de az\'{u}car, en particular, transportadores de glucosa SGLT dependientes de sodio son esenciales en la producci\'{o}n de metabolismo y la energ\'{i}a celular.Los cotransportadores SGLT son miembros de la familia de portadores de soluto (SLC5) y algunos de estos transportadores de inter\'{e}s tienen una secuencia y estructura similar tridimensional similar. En este caso se examin\'{o} el  co-transportador dependiente  sodio galactosa del Vibrio parahaemolyticus (vSGLT), que media el transporte de galactosa en el citoplasma de las bacterias Vibrio parahaemolyticus. Seg\'{u}n la literatura, la cin\'{e}tica del co-transportador tiene entre 5 y 6 estados o conformaciones, pero en este caso de que la desvinculaci\'{o}n de los sustratos se estudia la conformaci\'{o}n, tambi\'{e}n conocido como modelo de liberaci\'{o}n estado de co-transportador que mira hacia dentro. se realiz\'{o} un estudio computacional para analizar los movimientos globales de un transportador vSGLT, y comparamos nuestros resultados computacionales con los que se encuentran en los anteriores informes experimentales. an\'{a}lisis de modos normales con un modelo el\'{a}stico de red (ENM) fue utilizado para explorar los cambios en los movimientos globales entre vSGLT en la presencia o ausencia de los iones que transportan (Na +, galactosa). ENM se ha demostrado que es un c\'{a}lculo \'{u}til herramienta para predecir la din\'{a}mica de las prote\'{i}nas de membrana en muchas aplicaciones. los modos normales m\'{a}s bajas generadas por la ENM proporcionar informaci\'{o}n valiosa sobre la din\'{a}mica global de las biomol\'{e}culas que son relevantes para su funci\'{o}n.\\

\textbf{\small Palabras clave: Modos Normales, modelos de redes el\'{a}sticas, modelo de redes anisotr\'{o}picas, modelo de redes gaussianas, prote\'{i}na, cotransporte, simportador de sodio galactosa, vSGLT}.\\
\newpage
\textbf{\LARGE Abstract}\\\\
The study of sugar cotransporters, in particular sodium-dependent glucose transporters SGLTs are essential in cellular metabolism and energy production. SGLTs are members of solute carrier family (SLC5) and some of our interest transporters have similar sequence and also 3-dimensional structure. In this case we examined the sodium- dependent glucose co-transporter of Vibrio par- ahaemolyticus (vSGLT) which mediates galactose transport into the cytoplasm of vibrio parahaemolyticus bacteria. According literature, kinetic of co- transporter has between 5 and 6 states or conformations, but in this case the un- binding of substrates is studied by inward-facing conformation, also known as state release model of co-transporter. We performed a computational study to analyze the global movements of a vSGLT transporter, and we compared our computational results with those found in previous experimental reports. Normal modes analysis with an Elastic Network Model (ENM) was used to explore the changes in global movements between vSGLT in the presence or absence of the ions they transport (Na+, galactose). ENM has been shown to be a useful computational tool for predicting the dynamics of membrane pro- teins in many applications. The lowest normal modes generated by the ENM provide valuable insight into the global dynamics of biomolecules that are rele- vant to their function.\\[2.0cm]
\textbf{\small Keywords: Normal modes, elastic network models, anisotropic network model, Gaussian network model, protein, cotransporter, galactose sodium symporter, vSGLT}\\