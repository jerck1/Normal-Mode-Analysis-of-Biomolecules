\begin{appendix}
\chapter{Anexo A: Matriz de Masa del sistema }\label{AnexoA}
En tres dimensiones, como es nuestro caso real, la transformaci\'{o}n \eqref{eq:7} depender\'{a} del modelo escogido. Exceptuando el modelo de redes anisotropicas ANM, la transformaci\'{o}n de coordenadas va de $\mathbf{r}_{i}$ posiciones con $i=1,2,...,N$ ($3N$ coordenadas) a $q_j$ coordenadas $j=1,2,..,3N$:
\begin{equation*}
\mathbf{r}_{i}\longrightarrow q_{j}
\end{equation*}
Con
\begin{equation*}
i=1,2,...,N\mbox{  }j=1,2,..,3N
\end{equation*}
Por cada componente en cartesianas:
\begin{eqnarray}\label{eq:10}
\begin{array}{cccccc}
q_1=x_1-x_{10}&q_4=x_2-x_{20}&\cdots &q_{3i-2}=x_i-x_{i0}&\cdots &q_{3N-2}=x_N-x_{N0} \\
q_2=y_1-y_{10}&q_5=y_2-y_{20}&\cdots &q_{3i-1}=y_i-y_{i0}&\cdots &q_{3N-1}=y_N-y_{N0}\\
q_3=z_1-z_{10}&q_6=z_2-z_{20}&\cdots &q_{3i}=z_i-z_{i0}&\cdots &q_{3N}=z_N-z_{N0}\\
\end{array}
\end{eqnarray}
Para esta transformaci\'{o}n, \eqref{eq:7} se convierte en:
\begin{eqnarray}\label{eq:11}
M_{jk}&=&\sum_{i=1}^{N} m_{i}\left( \delta_{i,3j-2}\delta_{jk}+\delta_{i,3j-1}\delta_{jk}+  \delta_{i,3j}\delta_{jk}\right)\nonumber \\
M_{jk}&=&m_{3j-2}\delta_{jk}+m_{3j-1}\delta_{jk}+m_{3j}\delta_{jk} \nonumber \\
M_{jk}&=&\left( m_{3j-2}+m_{3j-1}+m_{3j} \right) \delta_{jk}
\end{eqnarray}
En \eqref{eq:11} debe resaltarse que para $j=k=3N$, el elemento de matriz $M_{3N,3N}$ requiere las masas $m_{3(3N)-2}=m_{9N-2}$, $m_{3(3N)-1}=m_{9N-1}$ y $m_{3(3N)}=m_{9N}$, sin embargo ! no hay $9N$ masas!, el n\'{u}mero de masas es el mismo n\'{u}mero de nodos: $N$, entonces, para poder calcular la matriz $\mathbf{M}$ es necesario definir lo siguiente:
\begin{equation}\label{eq:12}
m_{N+1},m_{N+2},...,m_{3N}=0
\end{equation}
Como a partir de $N+1$ las masas son nulas, la matriz de masa (que es diagonal) tiene elementos nulos si  
\begin{eqnarray*}
3j-2=N+1\\
j=\frac{N+3}{3} \\
\end{eqnarray*}
Como no siempre $N$ es m\'{u}ltiplo de 3, se escoge el entero menor que este m\'{a}s cerca al valor:
\begin{equation}\label{eq:13}
j=\left \lfloor\frac{N+3}{3}\right \rfloor
\end{equation}
Donde $\lfloor \rfloor$ representa la funci\'{o}n piso.

%\chapter{Anexo:}
\chapter{Anexo B: Fluctuaciones Pico}\label{AnexoB}
\subsubsection{ANM Previo}
En las tablas \ref{tab:flu1} y \ref{tab:flu2} se muestran los identificadores de residuo que presentan m\'{a}ximos o m\'{i}nimos locales (picos) en las figuras \ref{fig:ANM_pre1} \ref{fig:ANM_pre2} y \ref{fig:ANM_pre3}, se muestra por cada distancia de corte entre $8\AA$ y $14\AA$ y para cada modelo de la estructura aplicado (Ca, Ca$+$Na, Ca$+$Na$+$GalCa$+$Gal). Los factores b est\'{a}n normalizados a 1. \\
\begin{table}[ht]
\centering
\begin{adjustbox}{width=1.0\textwidth,center}
 \begin{tabular}[c]{|c|c|}
\multicolumn{2}{c}{$R_c=$8$\AA$}\\\hline
\textbf{Residuo}&\textbf{Factor B}\\\hline
\multicolumn{2}{c}{Con solo C$\alpha$}\\\hline
       253& 0.0416425\\
       251& 0.0415383\\
       249& 0.0377258\\
       479& 0.0376493\\
       250& 0.0373515\\\hline
\multicolumn{2}{c}{C$\alpha$ $+$ Gal}\\\hline
       254& 0.0923907\\
       250& 0.0911848\\
       251& 0.0889088\\
       252& 0.0886782\\
       253& 0.0860388\\\hline
\multicolumn{2}{c}{C$\alpha$ $+$ Gal $+$ Na}\\\hline
       254& 0.0925526\\
       250& 0.0913644\\
       251& 0.0890739\\
       252& 0.0888302\\
       253& 0.0861924\\\hline
\multicolumn{2}{c}{C$\alpha$ $+$ Na}\\\hline
       480&  0.110012\\
        88&  0.109922\\
       479&  0.109521\\
       476&  0.107543\\
       438&  0.106579\\\hline
\end{tabular}
\begin{tabular}[c]{|c|c|}
\multicolumn{2}{c}{$R_c=$9$\AA$}\\\hline
\textbf{Residuo}&\textbf{Factor B}\\\hline
\multicolumn{2}{c}{Con solo C$\alpha$}\\\hline
       253& 0.0416425\\
       251& 0.0415383\\
       249& 0.0377258\\
       479& 0.0376493\\
       250& 0.0373515\\\hline
\multicolumn{2}{c}{C$\alpha$ $+$ Gal}\\\hline
       475&  0.105822\\
       479&  0.105115\\
       250&  0.103229\\
       252&  0.101724\\
        88&  0.100704\\\hline
\multicolumn{2}{c}{C$\alpha$ $+$ Gal $+$ Na}\\\hline
       475&  0.106073\\
       479&  0.105362\\
       250&  0.103468\\
       252&  0.101948\\
        88&  0.100918\\\hline
\multicolumn{2}{c}{C$\alpha$ $+$ Na}\\\hline
       255&  0.116419\\
       257&  0.114597\\
       249&  0.112165\\
       438&  0.112136\\
       258&  0.105612\\\hline
\end{tabular}
\begin{tabular}[c]{|c|c|}
\multicolumn{2}{c}{$R_c=$10$\AA$}\\\hline
\textbf{Residuo}&\textbf{Factor B}\\\hline
\multicolumn{2}{c}{Con solo C$\alpha$}\\\hline
       253&  0.134993\\
       257&  0.133941\\
       258&  0.133659\\
       254&  0.131689\\
       255&  0.130529\\\hline
\multicolumn{2}{c}{C$\alpha$ $+$ Gal}\\\hline
       256&  0.126339\\
       255&  0.125446\\
       257&  0.125102\\
       249&  0.123929\\
       258&  0.117366\\\hline
\multicolumn{2}{c}{C$\alpha$ $+$ Gal $+$ Na}\\\hline
       256&  0.126633\\
       255&  0.125735\\
       257&  0.125394\\
       249&  0.124238\\
       258&   0.11696\\\hline
\multicolumn{2}{c}{C$\alpha$ $+$ Na}\\\hline
       255&  0.134158\\
       257&  0.132059\\
       249&  0.129256\\
       438&  0.129223\\
       258&  0.121705\\\hline
\end{tabular}
\begin{tabular}[c]{|c|c|}
\multicolumn{2}{c}{$R_c=$11$\AA$}\\\hline
\textbf{Residuo}&\textbf{Factor B}\\\hline
\multicolumn{2}{c}{Con solo C$\alpha$}\\\hline
       479& 0.0732422\\
       253& 0.0729239\\
       250& 0.0727664\\
       478&  0.071105\\
       251& 0.0709591\\\hline
\multicolumn{2}{c}{C$\alpha$ $+$ Gal}\\\hline
       250&  0.130337\\
       256&  0.129298\\
       255&  0.129017\\
       258&   0.12784\\
       257&  0.127207\\\hline
\multicolumn{2}{c}{C$\alpha$ $+$ Gal $+$ Na}\\\hline
       259&  0.130679\\
       256&  0.129646\\
       255&  0.128714\\
       258&  0.127578\\
       257&  0.127547\\\hline
\multicolumn{2}{c}{C$\alpha$ $+$ Na}\\\hline
       256&  0.134707\\
       259&  0.134242\\
       255&  0.133804\\
       258&  0.130805\\
       257&  0.130507\\\hline
\end{tabular}
\begin{tabular}[c]{|c|c|}
\multicolumn{2}{c}{$R_c=$12$\AA$}\\\hline
\textbf{Residuo}&\textbf{Factor B}\\\hline
\multicolumn{2}{c}{Con solo C$\alpha$}\\\hline
       479& 0.0643314\\
       253& 0.0640519\\
       250& 0.0639135\\
       478& 0.0624543\\
       251&  0.062326\\\hline
\multicolumn{2}{c}{C$\alpha$ $+$ Gal}\\\hline
       254&  0.115136\\
       253&  0.114893\\
       259&  0.114486\\
       252&  0.111882\\
       256&  0.110051\\\hline
\multicolumn{2}{c}{C$\alpha$ $+$ Gal $+$ Na}\\\hline
       250& 0.0780249\\
       251& 0.0762417\\
       254& 0.0747905\\
       252& 0.0742422\\
       253& 0.0728088\\\hline
\multicolumn{2}{c}{C$\alpha$ $+$ Na}\\\hline
       258&  0.115959\\
       253&  0.113907\\
       255&   0.11262\\
       256&   0.11215\\
       252&  0.111072\\\hline
\end{tabular}
\begin{tabular}[c]{|c|c|}
\multicolumn{2}{c}{$R_c=$13$\AA$}\\\hline
\textbf{Residuo}&\textbf{Factor B}\\\hline
\multicolumn{2}{c}{Con solo C$\alpha$}\\\hline
       478& 0.0980013\\
       252& 0.0977737\\
       250& 0.0951496\\
       480& 0.0945263\\
       479& 0.0905842\\\hline
\multicolumn{2}{c}{C$\alpha$ $+$ Gal}\\\hline
       258&    0.1166\\
       253&  0.114328\\
       255&  0.112683\\
       256&  0.112637\\
       252&  0.111443\\\hline
\multicolumn{2}{c}{C$\alpha$ $+$ Gal $+$ Na}\\\hline
       251& 0.0895345\\
       250& 0.0884175\\
       254& 0.0862215\\
       253& 0.0858509\\
       252& 0.0829043\\\hline
\multicolumn{2}{c}{C$\alpha$ $+$ Na}\\\hline
       256&  0.117632\\
       258&  0.117381\\
       255&  0.117372\\
       252&  0.116511\\
       253&   0.11622\\\hline
\end{tabular}
\begin{tabular}[c]{|c|c|}
\multicolumn{2}{c}{$R_c=$14$\AA$}\\\hline
\textbf{Residuo}&\textbf{Factor B}\\\hline
\multicolumn{2}{c}{Con solo C$\alpha$}\\\hline
       250&   0.10348\\
       476&  0.102362\\
       480&  0.101862\\
       253&  0.101347\\
       479& 0.0985975\\\hline
\multicolumn{2}{c}{C$\alpha$ $+$ Gal}\\\hline
       256&   0.11746\\
       258&   0.11726\\
       255&  0.117135\\
       252&  0.116212\\
       253&   0.11608\\\hline
\multicolumn{2}{c}{C$\alpha$ $+$ Gal $+$ Na}\\\hline
       251& 0.0895255\\
       250& 0.0884151\\
       254& 0.0862144\\
       253& 0.0858443\\
       252& 0.0828968\\\hline
\multicolumn{2}{c}{C$\alpha$ $+$ Na}\\\hline
       256&  0.117624\\
       258&  0.117373\\
       255&  0.117364\\
       252&  0.116503\\
       253&  0.116212\\\hline
\end{tabular}

 \end{adjustbox}
 \caption{Lista de los cinco n\'{u}meros de residuo que corresponden a los menores factores B}\label{tab:flu1}
\end{table}
\begin{table}[H]
\centering
\begin{adjustbox}{width=1.0\textwidth,center}
 \begin{tabular}[c]{|c|c|}
\multicolumn{2}{c}{$R_c=$8$\AA$}\\\hline
\textbf{Residuo}&\textbf{Factor B}\\\hline
\multicolumn{2}{c}{Con solo C$\alpha$}\\\hline
       543&  0.999999\\
       318&  0.798209\\
       319&   0.77496\\
       321&  0.741277\\
       317&  0.668767\\
\hline
\multicolumn{2}{c}{C$\alpha$ $+$ Gal}\\\hline
       209&  0.925647\\
       210&  0.879899\\
       318&  0.781582\\
       386&  0.775865\\
       384&  0.773399\\
\hline
\multicolumn{2}{c}{C$\alpha$ $+$ Gal $+$ Na}\\\hline
       209&  0.927167\\
       210&  0.881346\\
       318&  0.782798\\
       386&   0.77716\\
       384&  0.774713\\
\hline
\multicolumn{2}{c}{C$\alpha$ $+$ Na}\\\hline
       209&  0.868804\\
       210&  0.845608\\
       318&  0.827396\\
       211&  0.716527\\
       384&  0.702102\\
\hline
\end{tabular}
\begin{tabular}[c]{|c|c|}
\multicolumn{2}{c}{$R_c=$9$\AA$}\\\hline
\textbf{Residuo}&\textbf{Factor B}\\\hline
\multicolumn{2}{c}{Con solo C$\alpha$}\\\hline
       543&  0.999999\\
       318&  0.798209\\
       319&   0.77496\\
       321&  0.741277\\
       317&  0.668767\\
\hline
\multicolumn{2}{c}{C$\alpha$ $+$ Gal}\\\hline
       209&  0.882793\\
       210&  0.859023\\
       318&   0.84161\\
       211&  0.727772\\
       384&  0.714453\\
\hline
\multicolumn{2}{c}{C$\alpha$ $+$ Gal $+$ Na}\\\hline
       209&  0.884548\\
       210&  0.860736\\
       318&  0.843289\\
       211&  0.729196\\
       384&  0.715938\\
\hline
\multicolumn{2}{c}{C$\alpha$ $+$ Na}\\\hline
       210&  0.851038\\
       385&   0.83699\\
       209&  0.831604\\
       318&  0.789342\\
       211&  0.702455\\
\hline
\end{tabular}
\begin{tabular}[c]{|c|c|}
\multicolumn{2}{c}{$R_c=$10$\AA$}\\\hline
\textbf{Residuo}&\textbf{Factor B}\\\hline
\multicolumn{2}{c}{Con solo C$\alpha$}\\\hline
       209&   0.97897\\
       210&  0.971493\\
       318&  0.781321\\
       311&  0.755165\\
       315&  0.744612\\
\hline
\multicolumn{2}{c}{C$\alpha$ $+$ Gal}\\\hline
       210&  0.997761\\
       385&  0.982432\\
       209&  0.974967\\
       318&  0.926077\\
       211&  0.823312\\
\hline
\multicolumn{2}{c}{C$\alpha$ $+$ Gal $+$ Na}\\\hline
       210&  0.999999\\
       385&  0.984738\\
       209&  0.977157\\
       318&  0.928175\\
       211&  0.825091\\
\hline
\multicolumn{2}{c}{C$\alpha$ $+$ Na}\\\hline
       210&  0.980716\\
       385&  0.964528\\
       209&  0.958321\\
       318&  0.909619\\
       211&  0.809492\\
\hline
\end{tabular}
\begin{tabular}[c]{|c|c|}
\multicolumn{2}{c}{$R_c=$11$\AA$}\\\hline
\textbf{Residuo}&\textbf{Factor B}\\\hline
\multicolumn{2}{c}{Con solo C$\alpha$}\\\hline
       209&  0.857345\\
       386&  0.806943\\
       318&  0.796594\\
       388&  0.790143\\
       384&  0.737336\\
\hline
\multicolumn{2}{c}{C$\alpha$ $+$ Gal}\\\hline
       210&  0.997539\\
       209&  0.923375\\
       318&  0.866003\\
       382&  0.813186\\
       211&  0.803287\\
\hline
\multicolumn{2}{c}{C$\alpha$ $+$ Gal $+$ Na}\\\hline
       210&         1\\
       209&  0.925618\\
       318&  0.868137\\
       382&  0.815369\\
       211&  0.805206\\
\hline
\multicolumn{2}{c}{C$\alpha$ $+$ Na}\\\hline
       210&  0.978538\\
       209&   0.90573\\
       318&  0.848982\\
       382&  0.797229\\
       211&  0.788358\\
\hline
\end{tabular}
\begin{tabular}[c]{|c|c|}
\multicolumn{2}{c}{$R_c=$12$\AA$}\\\hline
\textbf{Residuo}&\textbf{Factor B}\\\hline
\multicolumn{2}{c}{Con solo C$\alpha$}\\\hline
       209&  0.753039\\
       386&  0.708769\\
       318&  0.699679\\
       388&  0.694012\\
       384&   0.64763\\
\hline
\multicolumn{2}{c}{C$\alpha$ $+$ Gal}\\\hline
       210&  0.877561\\
       209&  0.857573\\
       318&  0.737105\\
       211&  0.700336\\
       545&  0.691128\\
\hline
\multicolumn{2}{c}{C$\alpha$ $+$ Gal $+$ Na}\\\hline
       209&  0.999999\\
       318&  0.895038\\
       388&    0.7751\\
       343&  0.764006\\
       210&  0.752476\\
\hline
\multicolumn{2}{c}{C$\alpha$ $+$ Na}\\\hline
       210&   0.94773\\
       209&  0.821616\\
       318&  0.695025\\
       211&  0.676508\\
       545&  0.615122\\
\hline
\end{tabular}
\begin{tabular}[c]{|c|c|}
\multicolumn{2}{c}{$R_c=$13$\AA$}\\\hline
\textbf{Residuo}&\textbf{Factor B}\\\hline
\multicolumn{2}{c}{Con solo C$\alpha$}\\\hline
       209&  0.981844\\
       318&   0.89057\\
       210&  0.832029\\
       317&  0.726279\\
       314&  0.710232\\
\hline
\multicolumn{2}{c}{C$\alpha$ $+$ Gal}\\\hline
       210&  0.978828\\
       209&  0.848552\\
       318&  0.717871\\
       211&  0.698532\\
       545&  0.641043\\
\hline
\multicolumn{2}{c}{C$\alpha$ $+$ Gal $+$ Na}\\\hline
       209&  0.999999\\
       318&  0.907254\\
       210&  0.847078\\
       317&  0.739736\\
       314&  0.723279\\
\hline
\multicolumn{2}{c}{C$\alpha$ $+$ Na}\\\hline
       209&  0.889031\\
       210&  0.874584\\
       318&  0.696198\\
       211&  0.662066\\
       311&  0.656291\\
\hline
\end{tabular}
\begin{tabular}[c]{|c|c|}
\multicolumn{2}{c}{$R_c=$14$\AA$}\\\hline
\textbf{Residuo}&\textbf{Factor B}\\\hline
\multicolumn{2}{c}{Con solo C$\alpha$}\\\hline
       209&  0.947309\\
       210&  0.900641\\
       318&  0.799395\\
       386&  0.792957\\
       384&   0.79044\\
\hline
\multicolumn{2}{c}{C$\alpha$ $+$ Gal}\\\hline
       209&  0.908923\\
       210&  0.894141\\
       318&  0.711833\\
       211&  0.676765\\
       311&  0.670858\\
\hline
\multicolumn{2}{c}{C$\alpha$ $+$ Gal $+$ Na}\\\hline
       209&         1\\
       318&   0.90724\\
       210&   0.84708\\
       317&  0.739725\\
       314&  0.723268\\
\hline
\multicolumn{2}{c}{C$\alpha$ $+$ Na}\\\hline
       209&  0.888968\\
       210&  0.874523\\
       318&  0.696149\\
       211&   0.66202\\
       311&  0.656245\\
\hline
\end{tabular}

 \end{adjustbox}
  \caption{Lista de los cinco n\'{u}meros de residuo que corresponden a los mayores factores B}\label{tab:flu2}
\end{table}
\subsubsection{ANM con la Estructura Completa}
En las tablas \ref{tab:flu3} y \ref{tab:flu4} se muestran los identificadores de residuo que presentan m\'{a}ximos o m\'{i}nimos locales (picos) en las figuras \ref{fig:ANM_pos1} \ref{fig:ANM_pos2} y \ref{fig:ANM_pos3}, se muestra por cada distancia de corte entre $8\AA$ y $20\AA$ y para cada modelo de la estructura aplicado (Ca, Ca$+$Na, Ca$+$Na$+$GalCa$+$Gal). Los factores b est\'{a}n normalizados a 1. \\
\begin{table}[ht]
\centering
\begin{adjustbox}{width=1.0\textwidth,center}
 \begin{tabular}[c]{|c|c|}
\multicolumn{2}{c}{$R_c=$8$\AA$}\\\hline
\textbf{Residuo}&\textbf{Factor B}\\\hline
\multicolumn{2}{c}{Con solo C$\alpha$}\\\hline
        57& 0.0204681\\
       552&  0.020373\\
        52&  0.020369\\
       550& 0.0202855\\
        55&  0.019405\\\hline
\multicolumn{2}{c}{C$\alpha$ $+$ Gal}\\\hline
       238& 0.0168845\\
       559&  0.016211\\
        52& 0.0160602\\
       560& 0.0159309\\
        53& 0.0158506\\\hline
\multicolumn{2}{c}{C$\alpha$ $+$ Gal $+$ Na}\\\hline
       238& 0.0170197\\
       560& 0.0162525\\
        52& 0.0160965\\
       561& 0.0159908\\
        53& 0.0158986\\\hline
\multicolumn{2}{c}{C$\alpha$ $+$ Na}\\\hline
        57& 0.0206997\\
       553& 0.0206077\\
        52&  0.020523\\
       551& 0.0205108\\
        55& 0.0196425\\\hline
\end{tabular}
\begin{tabular}[c]{|c|c|}
\multicolumn{2}{c}{$R_c=$9$\AA$}\\\hline
\textbf{Residuo}&\textbf{Factor B}\\\hline
\multicolumn{2}{c}{Con solo C$\alpha$}\\\hline
        51& 0.0224673\\
       551& 0.0218607\\
        56& 0.0214665\\
       550& 0.0211568\\
        55& 0.0204383\\\hline
\multicolumn{2}{c}{C$\alpha$ $+$ Gal}\\\hline
       564& 0.0197437\\
        57& 0.0196957\\
        55& 0.0196049\\
       559& 0.0181753\\
        52& 0.0176356\\\hline
\multicolumn{2}{c}{C$\alpha$ $+$ Gal $+$ Na}\\\hline
       565& 0.0199296\\
        57& 0.0198849\\
        55& 0.0197845\\
       560& 0.0183113\\
        52& 0.0177768\\\hline
\multicolumn{2}{c}{C$\alpha$ $+$ Na}\\\hline
        51& 0.0222457\\
       552& 0.0220904\\
        56& 0.0217036\\
       551& 0.0213413\\
        55& 0.0206398\\\hline
\end{tabular}
\begin{tabular}[c]{|c|c|}
\multicolumn{2}{c}{$R_c=$10$\AA$}\\\hline
\textbf{Residuo}&\textbf{Factor B}\\\hline
\multicolumn{2}{c}{Con solo C$\alpha$}\\\hline
        55& 0.0707494\\
        56& 0.0704414\\
       553& 0.0686513\\
        52& 0.0683517\\
        58& 0.0668804\\\hline
\multicolumn{2}{c}{C$\alpha$ $+$ Gal}\\\hline
        53& 0.0635874\\
       564& 0.0633834\\
        57& 0.0628706\\
       559& 0.0569826\\
        52& 0.0547921\\\hline
\multicolumn{2}{c}{C$\alpha$ $+$ Gal $+$ Na}\\\hline
        53& 0.0637438\\
       565& 0.0636418\\
        57& 0.0631314\\
       560& 0.0571122\\
        52& 0.0549317\\\hline
\multicolumn{2}{c}{C$\alpha$ $+$ Na}\\\hline
       548& 0.0708839\\
        56& 0.0707042\\
       554& 0.0689513\\
        52& 0.0685161\\
        58& 0.0671936\\\hline
\end{tabular}
\begin{tabular}[c]{|c|c|}
\multicolumn{2}{c}{$R_c=$11$\AA$}\\\hline
\textbf{Residuo}&\textbf{Factor B}\\\hline
\multicolumn{2}{c}{Con solo C$\alpha$}\\\hline
        55& 0.0854032\\
       553&  0.085272\\
        58& 0.0852603\\
       552& 0.0840119\\
        57& 0.0835917\\\hline
\multicolumn{2}{c}{C$\alpha$ $+$ Gal}\\\hline
        54& 0.0775723\\
       559& 0.0746854\\
        52& 0.0735683\\
       564& 0.0722947\\
        57& 0.0720476\\\hline
\multicolumn{2}{c}{C$\alpha$ $+$ Gal $+$ Na}\\\hline
        54& 0.0766025\\
       560& 0.0733634\\
       565&  0.072488\\
        57& 0.0722296\\
        52& 0.0722008\\\hline
\multicolumn{2}{c}{C$\alpha$ $+$ Na}\\\hline
        55& 0.0855987\\
       554& 0.0855169\\
        58& 0.0855092\\
       553& 0.0842182\\
        57& 0.0837868\\\hline
\end{tabular}
\begin{tabular}[c]{|c|c|}
\multicolumn{2}{c}{$R_c=$12$\AA$}\\\hline
\textbf{Residuo}&\textbf{Factor B}\\\hline
\multicolumn{2}{c}{Con solo C$\alpha$}\\\hline
        58& 0.0810295\\
       552&  0.080928\\
       134&  0.080509\\
        54&  0.079496\\
       133&  0.077735\\\hline
\multicolumn{2}{c}{C$\alpha$ $+$ Gal}\\\hline
        54& 0.0698048\\
        57& 0.0693897\\
       564& 0.0686263\\
       560& 0.0637402\\
        53& 0.0629452\\\hline
\multicolumn{2}{c}{C$\alpha$ $+$ Gal $+$ Na}\\\hline
        54& 0.0673207\\
       560&  0.064474\\
       565& 0.0637047\\
        57& 0.0634776\\
        52& 0.0634523\\\hline
\multicolumn{2}{c}{C$\alpha$ $+$ Na}\\\hline
        53& 0.0807021\\
       134& 0.0805023\\
       130& 0.0799806\\
        54& 0.0785719\\
       133& 0.0778298\\\hline
\end{tabular}
\begin{tabular}[c]{|c|c|}
\multicolumn{2}{c}{$R_c=$13$\AA$}\\\hline
\textbf{Residuo}&\textbf{Factor B}\\\hline
\multicolumn{2}{c}{Con solo C$\alpha$}\\\hline
       317& 0.0833098\\
       550& 0.0832913\\
        55& 0.0815015\\
       549& 0.0813637\\
        54& 0.0794106\\\hline
\multicolumn{2}{c}{C$\alpha$ $+$ Gal}\\\hline
       578& 0.0728529\\
       560&  0.072808\\
        53&  0.072133\\
       561& 0.0708486\\
        54&   0.06985\\\hline
\multicolumn{2}{c}{C$\alpha$ $+$ Gal $+$ Na}\\\hline
        57& 0.0693383\\
        54& 0.0686569\\
       565& 0.0685766\\
       561& 0.0628526\\
        53& 0.0620548\\\hline
\multicolumn{2}{c}{C$\alpha$ $+$ Na}\\\hline
       317& 0.0834629\\
       551& 0.0823612\\
       550& 0.0806418\\
        55& 0.0805926\\
        54& 0.0787186\\\hline
\end{tabular}
\begin{tabular}[c]{|c|c|}
\multicolumn{2}{c}{$R_c=$14$\AA$}\\\hline
\textbf{Residuo}&\textbf{Factor B}\\\hline
\multicolumn{2}{c}{Con solo C$\alpha$}\\\hline
        53& 0.0892839\\
       549& 0.0829586\\
       550& 0.0823541\\
        54&  0.081156\\
        55& 0.0807122\\\hline
\multicolumn{2}{c}{C$\alpha$ $+$ Gal}\\\hline
       560& 0.0766249\\
        53& 0.0759543\\
        55& 0.0754406\\
       561& 0.0731713\\
        54& 0.0721913\\\hline
\multicolumn{2}{c}{C$\alpha$ $+$ Gal $+$ Na}\\\hline
       561& 0.0759368\\
        53& 0.0752674\\
        55& 0.0748615\\
       562& 0.0726109\\
        54& 0.0716514\\\hline
\multicolumn{2}{c}{C$\alpha$ $+$ Na}\\\hline
        53& 0.0885695\\
       550& 0.0823812\\
       551& 0.0817084\\
        54& 0.0806019\\
        55&  0.080084\\\hline
\end{tabular}
\begin{tabular}[c]{|c|c|}
\multicolumn{2}{c}{$R_c=$15$\AA$}\\\hline
\textbf{Residuo}&\textbf{Factor B}\\\hline
\multicolumn{2}{c}{Con solo C$\alpha$}\\\hline
        53& 0.0892606\\
       549& 0.0885579\\
        54& 0.0870733\\
       550& 0.0868694\\
        55& 0.0854985\\\hline
\multicolumn{2}{c}{C$\alpha$ $+$ Gal}\\\hline
        53& 0.0791361\\
       561& 0.0790233\\
       562& 0.0784417\\
        55& 0.0777231\\
        54& 0.0777177\\\hline
\multicolumn{2}{c}{C$\alpha$ $+$ Gal $+$ Na}\\\hline
       561& 0.0759368\\
        53& 0.0752674\\
        55& 0.0748615\\
       562& 0.0726109\\
        54& 0.0716514\\\hline
\multicolumn{2}{c}{C$\alpha$ $+$ Na}\\\hline
        53& 0.0887444\\
       550& 0.0880371\\
        54& 0.0865809\\
       551& 0.0862878\\
        55& 0.0849419\\\hline
\end{tabular}
\begin{tabular}[c]{|c|c|}
\multicolumn{2}{c}{$R_c=$16$\AA$}\\\hline
\textbf{Residuo}&\textbf{Factor B}\\\hline
\multicolumn{2}{c}{Con solo C$\alpha$}\\\hline
        54& 0.0857322\\
       550& 0.0851815\\
       548& 0.0842129\\
        55& 0.0840389\\
        53& 0.0836223\\\hline
\multicolumn{2}{c}{C$\alpha$ $+$ Gal}\\\hline
        54& 0.0782085\\
       562& 0.0776999\\
       560& 0.0774717\\
        53& 0.0770041\\
        55& 0.0766522\\\hline
\multicolumn{2}{c}{C$\alpha$ $+$ Gal $+$ Na}\\\hline
        53& 0.0758611\\
       562& 0.0757707\\
       563& 0.0752099\\
        55&  0.074537\\
        54& 0.0745354\\\hline
\multicolumn{2}{c}{C$\alpha$ $+$ Na}\\\hline
        54& 0.0853265\\
       551&  0.084711\\
       549& 0.0838003\\
        55& 0.0835829\\
        53& 0.0832137\\\hline
\end{tabular}
\begin{tabular}[c]{|c|c|}
\multicolumn{2}{c}{$R_c=$17$\AA$}\\\hline
\textbf{Residuo}&\textbf{Factor B}\\\hline
\multicolumn{2}{c}{Con solo C$\alpha$}\\\hline
       547& 0.0853375\\
        51& 0.0852534\\
        52& 0.0847917\\
       548& 0.0841381\\
        53& 0.0836471\\\hline
\multicolumn{2}{c}{C$\alpha$ $+$ Gal}\\\hline
        52& 0.0794492\\
       558& 0.0794004\\
       560& 0.0788308\\
        53& 0.0784453\\
        51& 0.0782466\\\hline
\multicolumn{2}{c}{C$\alpha$ $+$ Gal $+$ Na}\\\hline
       559& 0.0789938\\
        52& 0.0789772\\
       561& 0.0784782\\
        53& 0.0780998\\
        51& 0.0778611\\\hline
\multicolumn{2}{c}{C$\alpha$ $+$ Na}\\\hline
       549& 0.0844185\\
       548& 0.0842797\\
        52& 0.0838335\\
        50& 0.0837296\\
        51& 0.0829022\\\hline
\end{tabular}
\begin{tabular}[c]{|c|c|}
\multicolumn{2}{c}{$R_c=$18$\AA$}\\\hline
\textbf{Residuo}&\textbf{Factor B}\\\hline
\multicolumn{2}{c}{Con solo C$\alpha$}\\\hline
       548& 0.0764402\\
       547&  0.076388\\
        52& 0.0759836\\
        50& 0.0758221\\
        51& 0.0750441\\\hline
\multicolumn{2}{c}{C$\alpha$ $+$ Gal}\\\hline
       559& 0.0724359\\
        52& 0.0720959\\
       558&  0.071894\\
        50& 0.0713848\\
        51& 0.0701683\\\hline
\multicolumn{2}{c}{C$\alpha$ $+$ Gal $+$ Na}\\\hline
       559& 0.0712774\\
        52& 0.0712624\\
       561& 0.0708121\\
        53& 0.0704707\\
        51& 0.0702553\\\hline
\multicolumn{2}{c}{C$\alpha$ $+$ Na}\\\hline
       549& 0.0761722\\
       548& 0.0760469\\
        52& 0.0756443\\
        50& 0.0755505\\
        51&  0.074804\\\hline
\end{tabular}
\begin{tabular}[c]{|c|c|}
\multicolumn{2}{c}{$R_c=$19$\AA$}\\\hline
\textbf{Residuo}&\textbf{Factor B}\\\hline
\multicolumn{2}{c}{Con solo C$\alpha$}\\\hline
       548& 0.0709647\\
        53& 0.0707113\\
        51& 0.0695061\\
       547& 0.0686808\\
        52& 0.0684025\\\hline
\multicolumn{2}{c}{C$\alpha$ $+$ Gal}\\\hline
        53&  0.068177\\
        50&  0.067962\\
        51& 0.0661747\\
       559& 0.0661545\\
        52&  0.065916\\\hline
\multicolumn{2}{c}{C$\alpha$ $+$ Gal $+$ Na}\\\hline
       560&  0.066809\\
        52& 0.0664972\\
       559& 0.0663817\\
        50& 0.0659505\\
        51& 0.0648034\\\hline
\multicolumn{2}{c}{C$\alpha$ $+$ Na}\\\hline
       549& 0.0707796\\
        53& 0.0705282\\
        51& 0.0693293\\
       548& 0.0684251\\
        52& 0.0681494\\\hline
\end{tabular}
\begin{tabular}[c]{|c|c|}
\multicolumn{2}{c}{$R_c=$20$\AA$}\\\hline
\textbf{Residuo}&\textbf{Factor B}\\\hline
\multicolumn{2}{c}{Con solo C$\alpha$}\\\hline
       548& 0.0727684\\
       547&   0.07024\\
        51&  0.070043\\
        52& 0.0700025\\
        50& 0.0698384\\\hline
\multicolumn{2}{c}{C$\alpha$ $+$ Gal}\\\hline
       558& 0.0700424\\
       559& 0.0683676\\
        52& 0.0681604\\
        51&  0.067409\\
        50& 0.0673795\\\hline
\multicolumn{2}{c}{C$\alpha$ $+$ Gal $+$ Na}\\\hline
       559& 0.0698707\\
       560& 0.0681607\\
        52& 0.0679558\\
        50& 0.0672642\\
        51& 0.0672586\\\hline
\multicolumn{2}{c}{C$\alpha$ $+$ Na}\\\hline
       549& 0.0726093\\
       548& 0.0700497\\
        51& 0.0699099\\
        52& 0.0698152\\
        50& 0.0697069\\\hline
\end{tabular}

 \end{adjustbox}
 \caption{Lista de los cinco n\'{u}meros de residuo que corresponden a los menores factores B}\label{tab:flu3}
\end{table}
\begin{table}[H]
\centering
\begin{adjustbox}{width=1.0\textwidth,center}
 \begin{tabular}[c]{|c|c|}
\multicolumn{2}{c}{$R_c=$8$\AA$}\\\hline
\textbf{Residuo}&\textbf{Factor B}\\\hline
\multicolumn{2}{c}{Con solo C$\alpha$}\\\hline
        30&  0.971744\\
        29&  0.879407\\
        28&  0.724005\\
        31&  0.676388\\
       526&  0.663871\\
\hline
\multicolumn{2}{c}{C$\alpha$ $+$ Gal}\\\hline
        30&         1\\
        29&  0.904975\\
        28&  0.744449\\
        31&   0.69698\\
       538&  0.684074\\
\hline
\multicolumn{2}{c}{C$\alpha$ $+$ Gal $+$ Na}\\\hline
        30&   0.99875\\
        29&  0.903707\\
        28&  0.740941\\
        31&  0.706648\\
       539&  0.693673\\
\hline
\multicolumn{2}{c}{C$\alpha$ $+$ Na}\\\hline
        30&    0.9719\\
        29&   0.87943\\
        28&  0.721631\\
        31&  0.686563\\
       527&  0.673962\\
\hline
\end{tabular}
\begin{tabular}[c]{|c|c|}
\multicolumn{2}{c}{$R_c=$9$\AA$}\\\hline
\textbf{Residuo}&\textbf{Factor B}\\\hline
\multicolumn{2}{c}{Con solo C$\alpha$}\\\hline
        31&  0.960466\\
       526&  0.939348\\
        30&  0.338672\\
       163&  0.330955\\
       301&  0.306698\\
\hline
\multicolumn{2}{c}{C$\alpha$ $+$ Gal}\\\hline
        31&   0.98799\\
       538&  0.966232\\
        30&  0.347147\\
       163&   0.33462\\
       301&    0.3149\\
\hline
\multicolumn{2}{c}{C$\alpha$ $+$ Gal $+$ Na}\\\hline
        31&         1\\
       539&  0.978003\\
        30&  0.347289\\
       301&  0.318753\\
       163&  0.309178\\
\hline
\multicolumn{2}{c}{C$\alpha$ $+$ Na}\\\hline
        31&  0.973131\\
       527&  0.951758\\
        30&   0.33908\\
       301&  0.310733\\
       163&  0.303916\\
\hline
\end{tabular}
\begin{tabular}[c]{|c|c|}
\multicolumn{2}{c}{$R_c=$10$\AA$}\\\hline
\textbf{Residuo}&\textbf{Factor B}\\\hline
\multicolumn{2}{c}{Con solo C$\alpha$}\\\hline
        30&  0.976301\\
       370&  0.943922\\
       369&  0.880926\\
       371&  0.851762\\
        29&  0.807952\\
\hline
\multicolumn{2}{c}{C$\alpha$ $+$ Gal}\\\hline
        30&  0.997791\\
       370&  0.967282\\
       369&  0.902936\\
       371&  0.872108\\
       367&   0.82862\\
\hline
\multicolumn{2}{c}{C$\alpha$ $+$ Gal $+$ Na}\\\hline
        30&  0.999998\\
       370&  0.968786\\
       369&  0.904331\\
       371&  0.872344\\
       367&  0.830838\\
\hline
\multicolumn{2}{c}{C$\alpha$ $+$ Na}\\\hline
        30&  0.978882\\
       370&  0.945794\\
       369&    0.8827\\
       371&  0.852352\\
       367&  0.810441\\
\hline
\end{tabular}
\begin{tabular}[c]{|c|c|}
\multicolumn{2}{c}{$R_c=$11$\AA$}\\\hline
\textbf{Residuo}&\textbf{Factor B}\\\hline
\multicolumn{2}{c}{Con solo C$\alpha$}\\\hline
       192&  0.974512\\
       369&  0.945604\\
        30&  0.940945\\
        29&  0.932098\\
       371&  0.929013\\
\hline
\multicolumn{2}{c}{C$\alpha$ $+$ Gal}\\\hline
       192&    0.9963\\
       369&  0.969021\\
        30&  0.960716\\
        29&   0.95225\\
       371&   0.95153\\
\hline
\multicolumn{2}{c}{C$\alpha$ $+$ Gal $+$ Na}\\\hline
       192&  0.999998\\
       369&  0.970707\\
        30&  0.960954\\
        29&  0.953925\\
       371&   0.95286\\
\hline
\multicolumn{2}{c}{C$\alpha$ $+$ Na}\\\hline
       192&  0.978309\\
       369&  0.947543\\
        30&  0.941278\\
        29&  0.933905\\
       371&  0.930613\\
\hline
\end{tabular}
\begin{tabular}[c]{|c|c|}
\multicolumn{2}{c}{$R_c=$12$\AA$}\\\hline
\textbf{Residuo}&\textbf{Factor B}\\\hline
\multicolumn{2}{c}{Con solo C$\alpha$}\\\hline
       192&  0.977536\\
        30&   0.89929\\
       371&  0.897518\\
       301&  0.809255\\
       369&  0.801831\\
\hline
\multicolumn{2}{c}{C$\alpha$ $+$ Gal}\\\hline
       192&  0.999999\\
        30&  0.920047\\
       371&  0.919484\\
       301&  0.828482\\
       369&  0.821603\\
\hline
\multicolumn{2}{c}{C$\alpha$ $+$ Gal $+$ Na}\\\hline
       192&   0.87883\\
       369&  0.853087\\
        30&  0.844516\\
        29&  0.838339\\
       371&  0.837403\\
\hline
\multicolumn{2}{c}{C$\alpha$ $+$ Na}\\\hline
       192&  0.980853\\
        30&  0.900952\\
       371&  0.899573\\
       301&  0.812077\\
       369&  0.803678\\
\hline
\end{tabular}
\begin{tabular}[c]{|c|c|}
\multicolumn{2}{c}{$R_c=$13$\AA$}\\\hline
\textbf{Residuo}&\textbf{Factor B}\\\hline
\multicolumn{2}{c}{Con solo C$\alpha$}\\\hline
       192&  0.877689\\
       371&    0.8492\\
       369&  0.820846\\
        29&  0.803562\\
       367&  0.775953\\
\hline
\multicolumn{2}{c}{C$\alpha$ $+$ Gal}\\\hline
       192&  0.896496\\
       371&  0.868903\\
       369&  0.840064\\
        29&  0.821032\\
       367&  0.794193\\
\hline
\multicolumn{2}{c}{C$\alpha$ $+$ Gal $+$ Na}\\\hline
       192&  0.999998\\
        30&   0.91866\\
       371&  0.918428\\
       301&  0.828547\\
       369&   0.82068\\
\hline
\multicolumn{2}{c}{C$\alpha$ $+$ Na}\\\hline
       192&  0.879703\\
       371&  0.850815\\
       369&    0.8225\\
        29&  0.804918\\
       367&  0.777608\\
\hline
\end{tabular}
\begin{tabular}[c]{|c|c|}
\multicolumn{2}{c}{$R_c=$14$\AA$}\\\hline
\textbf{Residuo}&\textbf{Factor B}\\\hline
\multicolumn{2}{c}{Con solo C$\alpha$}\\\hline
       369&  0.975966\\
       371&   0.91383\\
       367&  0.903075\\
       192&  0.845035\\
       193&  0.807232\\
\hline
\multicolumn{2}{c}{C$\alpha$ $+$ Gal}\\\hline
       369&  0.997843\\
       371&  0.934174\\
       367&  0.923332\\
       192&  0.861854\\
       193&  0.823164\\
\hline
\multicolumn{2}{c}{C$\alpha$ $+$ Gal $+$ Na}\\\hline
       369&         1\\
       371&  0.936036\\
       367&  0.925423\\
       192&  0.863795\\
       193&  0.825009\\
\hline
\multicolumn{2}{c}{C$\alpha$ $+$ Na}\\\hline
       369&   0.97792\\
       371&  0.915506\\
       367&  0.904972\\
       192&  0.846781\\
       193&  0.808892\\
\hline
\end{tabular}
\begin{tabular}[c]{|c|c|}
\multicolumn{2}{c}{$R_c=$15$\AA$}\\\hline
\textbf{Residuo}&\textbf{Factor B}\\\hline
\multicolumn{2}{c}{Con solo C$\alpha$}\\\hline
       367&  0.952573\\
       369&  0.947907\\
       371&  0.914559\\
        29&  0.878665\\
       368&  0.771469\\
\hline
\multicolumn{2}{c}{C$\alpha$ $+$ Gal}\\\hline
       367&  0.973953\\
       369&  0.969079\\
       371&  0.934867\\
        29&  0.897946\\
       368&  0.788595\\
\hline
\multicolumn{2}{c}{C$\alpha$ $+$ Gal $+$ Na}\\\hline
       369&         1\\
       371&  0.936036\\
       367&  0.925423\\
       192&  0.863795\\
       193&  0.825009\\
\hline
\multicolumn{2}{c}{C$\alpha$ $+$ Na}\\\hline
       367&  0.954619\\
       369&   0.94974\\
       371&  0.916192\\
        29&  0.880418\\
       368&  0.772907\\
\hline
\end{tabular}
\begin{tabular}[c]{|c|c|}
\multicolumn{2}{c}{$R_c=$16$\AA$}\\\hline
\textbf{Residuo}&\textbf{Factor B}\\\hline
\multicolumn{2}{c}{Con solo C$\alpha$}\\\hline
       367&  0.978572\\
       369&  0.957639\\
       371&  0.911036\\
       368&  0.759721\\
       192&   0.74007\\
\hline
\multicolumn{2}{c}{C$\alpha$ $+$ Gal}\\\hline
       367&  0.999998\\
       369&  0.978491\\
       371&   0.93073\\
       368&  0.776157\\
       192&  0.754544\\
\hline
\multicolumn{2}{c}{C$\alpha$ $+$ Gal $+$ Na}\\\hline
       367&  0.942055\\
       369&  0.937138\\
       371&  0.903917\\
        29&  0.868361\\
       368&  0.762548\\
\hline
\multicolumn{2}{c}{C$\alpha$ $+$ Na}\\\hline
       367&  0.980836\\
       369&  0.959679\\
       371&  0.912762\\
       368&  0.761301\\
       192&  0.741862\\
\hline
\end{tabular}
\begin{tabular}[c]{|c|c|}
\multicolumn{2}{c}{$R_c=$17$\AA$}\\\hline
\textbf{Residuo}&\textbf{Factor B}\\\hline
\multicolumn{2}{c}{Con solo C$\alpha$}\\\hline
       369&  0.976494\\
       367&  0.953563\\
       371&  0.871317\\
       368&  0.822244\\
       370&   0.70725\\
\hline
\multicolumn{2}{c}{C$\alpha$ $+$ Gal}\\\hline
       369&  0.997516\\
       367&  0.974213\\
       371&  0.889948\\
       368&  0.839892\\
       370&  0.722351\\
\hline
\multicolumn{2}{c}{C$\alpha$ $+$ Gal $+$ Na}\\\hline
       369&         1\\
       367&   0.97675\\
       371&  0.891918\\
       368&  0.841971\\
       370&  0.723982\\
\hline
\multicolumn{2}{c}{C$\alpha$ $+$ Na}\\\hline
       369&   1.08729\\
       367&   1.01982\\
       371&  0.900285\\
       368&  0.898477\\
       370&  0.744277\\
\hline
\end{tabular}
\begin{tabular}[c]{|c|c|}
\multicolumn{2}{c}{$R_c=$18$\AA$}\\\hline
\textbf{Residuo}&\textbf{Factor B}\\\hline
\multicolumn{2}{c}{Con solo C$\alpha$}\\\hline
       369&  0.978463\\
       367&  0.917663\\
       371&  0.810386\\
       368&  0.808557\\
       370&   0.66992\\
\hline
\multicolumn{2}{c}{C$\alpha$ $+$ Gal}\\\hline
       369&  0.999999\\
       367&  0.938033\\
       371&  0.827844\\
       368&  0.826306\\
       370&  0.684326\\
\hline
\multicolumn{2}{c}{C$\alpha$ $+$ Gal $+$ Na}\\\hline
       369&  0.902317\\
       367&  0.881337\\
       371&  0.804792\\
       368&  0.759723\\
       370&   0.65326\\
\hline
\multicolumn{2}{c}{C$\alpha$ $+$ Na}\\\hline
       369&  0.981082\\
       367&  0.920197\\
       371&  0.812342\\
       368&   0.81071\\
       370&  0.671573\\
\hline
\end{tabular}
\begin{tabular}[c]{|c|c|}
\multicolumn{2}{c}{$R_c=$19$\AA$}\\\hline
\textbf{Residuo}&\textbf{Factor B}\\\hline
\multicolumn{2}{c}{Con solo C$\alpha$}\\\hline
       369&  0.977415\\
       368&  0.806767\\
       371&   0.77344\\
       367&  0.699925\\
       370&  0.687809\\
\hline
\multicolumn{2}{c}{C$\alpha$ $+$ Gal}\\\hline
       369&         1\\
       368&  0.825398\\
       371&  0.790453\\
       367&  0.716281\\
       370&  0.703006\\
\hline
\multicolumn{2}{c}{C$\alpha$ $+$ Gal $+$ Na}\\\hline
       369&  0.929389\\
       367&  0.871866\\
       371&  0.769197\\
       368&   0.76795\\
       370&  0.635883\\
\hline
\multicolumn{2}{c}{C$\alpha$ $+$ Na}\\\hline
       369&  0.980197\\
       368&  0.809063\\
       371&  0.775531\\
       367&  0.701981\\
       370&  0.689696\\
\hline
\end{tabular}
\begin{tabular}[c]{|c|c|}
\multicolumn{2}{c}{$R_c=$20$\AA$}\\\hline
\textbf{Residuo}&\textbf{Factor B}\\\hline
\multicolumn{2}{c}{Con solo C$\alpha$}\\\hline
       369&  0.974441\\
       371&   0.81452\\
       368&  0.769906\\
       367&   0.72264\\
       372&  0.628035\\
\hline
\multicolumn{2}{c}{C$\alpha$ $+$ Gal}\\\hline
       369&  0.997034\\
       371&  0.833353\\
       368&   0.78784\\
       367&  0.739628\\
       372&  0.642612\\
\hline
\multicolumn{2}{c}{C$\alpha$ $+$ Gal $+$ Na}\\\hline
       369&  0.999999\\
       371&  0.835806\\
       368&  0.790172\\
       367&  0.741854\\
       372&  0.644488\\
\hline
\multicolumn{2}{c}{C$\alpha$ $+$ Na}\\\hline
       369&  0.977421\\
       371&  0.816988\\
       368&  0.772251\\
       367&  0.724883\\
       372&  0.629921\\
\hline
\end{tabular}

 \end{adjustbox}
  \caption{Lista de los cinco n\'{u}meros de residuo que corresponden a los mayores factores B}\label{tab:flu4}
\end{table}

\end{appendix}
