\begin{appendix}
\chapter{Anexo 1: Matriz de Masa del sistema }\label{AnexoA}
En tres dimensiones, como es nuestro caso real, la transformaci\'{o}n \eqref{eq:7} depender\'{a} del modelo escogido. Exceptuando el modelo de redes gaussianas (Gaussian Network Model que por sus siglas en ingl\'{e}s es GNM) la transformaci\'{o}n de coordenadas va de $\mathbf{r}_{i}$ posiciones con $i=1,2,...,N$ ($3N$ coordenadas) a $q_j$ coordenadas $j=1,2,..,3N$:
\begin{equation*}
\mathbf{r}_{i}\longrightarrow q_{j}
\end{equation*}
Con
\begin{equation*}
i=1,2,...,N\mbox{  }j=1,2,..,3N
\end{equation*}
Por cada componente en cartesianas:
\begin{eqnarray}\label{eq:10}
\begin{array}{cccccc}
q_1=x_1-x_{10}&q_4=x_2-x_{20}&\cdots &q_{3i-2}=x_i-x_{i0}&\cdots &q_{3N-2}=x_N-x_{N0} \\
q_2=y_1-y_{10}&q_5=y_2-y_{20}&\cdots &q_{3i-1}=y_i-y_{i0}&\cdots &q_{3N-1}=y_N-y_{N0}\\
q_3=z_1-z_{10}&q_6=z_2-z_{20}&\cdots &q_{3i}=z_i-z_{i0}&\cdots &q_{3N}=z_N-z_{N0}\\
\end{array}
\end{eqnarray}
Para esta transformaci\'{o}n, \eqref{eq:7} se convierte en:
\begin{eqnarray}\label{eq:11}
M_{jk}&=&\sum_{i=1}^{N} m_{i}\left( \delta_{i,3j-2}\delta_{jk}+\delta_{i,3j-1}\delta_{jk}+  \delta_{i,3j}\delta_{jk}\right)\nonumber \\
M_{jk}&=&m_{3j-2}\delta_{jk}+m_{3j-1}\delta_{jk}+m_{3j}\delta_{jk} \nonumber \\
M_{jk}&=&\left( m_{3j-2}+m_{3j-1}+m_{3j} \right) \delta_{jk}
\end{eqnarray}
En \eqref{eq:11} debe resaltarse que para $j=k=3N$, el elemento de matriz $M_{3N,3N}$ requiere las masas $m_{3(3N)-2}=m_{9N-2}$, $m_{3(3N)-1}=m_{9N-1}$ y $m_{3(3N)}=m_{9N}$, sin embargo ! no hay $9N$ masas!, el n\'{u}mero de masas es el mismo n\'{u}mero de nodos: $N$, entonces, para poder calcular la matriz $\mathbf{M}$ es necesario definir lo siguiente:
\begin{equation}\label{eq:12}
m_{N+1},m_{N+2},...,m_{3N}=0
\end{equation}
Como a partir de $N+1$ las masas son nulas, la matriz de masa (que es diagonal) tiene elementos nulos si  
\begin{eqnarray*}
3j-2=N+1\\
j=\frac{N+3}{3} \\
\end{eqnarray*}
Como no siempre $N$ es m\'{u}ltiplo de 3, se escoge el entero menor que este m\'{a}s cerca al valor:
\begin{equation}\label{eq:13}
j=\left \lfloor\frac{N+3}{3}\right \rfloor
\end{equation}
Donde $\lfloor \rfloor$ representa la funci\'{o}n piso.

\chapter{Anexo:}


\end{appendix}
