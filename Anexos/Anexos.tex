\begin{appendix}
\chapter{Anexo A: Matriz de Masa del sistema }\label{AnexoA}
En tres dimensiones, como es nuestro caso real, la transformaci\'{o}n \eqref{eq:7} depender\'{a} del modelo escogido. Exceptuando el modelo de redes anisotropicas ANM, la transformaci\'{o}n de coordenadas va de $\mathbf{r}_{i}$ posiciones con $i=1,2,...,N$ ($3N$ coordenadas) a $q_j$ coordenadas $j=1,2,..,3N$:
\begin{equation*}
\mathbf{r}_{i}\longrightarrow q_{j}
\end{equation*}
Con
\begin{equation*}
i=1,2,...,N\mbox{  }j=1,2,..,3N
\end{equation*}
Por cada componente en cartesianas:
\begin{eqnarray}\label{eq:10}
\begin{array}{cccccc}
q_1=x_1-x_{10}&q_4=x_2-x_{20}&\cdots &q_{3i-2}=x_i-x_{i0}&\cdots &q_{3N-2}=x_N-x_{N0} \\
q_2=y_1-y_{10}&q_5=y_2-y_{20}&\cdots &q_{3i-1}=y_i-y_{i0}&\cdots &q_{3N-1}=y_N-y_{N0}\\
q_3=z_1-z_{10}&q_6=z_2-z_{20}&\cdots &q_{3i}=z_i-z_{i0}&\cdots &q_{3N}=z_N-z_{N0}\\
\end{array}
\end{eqnarray}
Para esta transformaci\'{o}n, \eqref{eq:7} se convierte en:
\begin{eqnarray}\label{eq:11}
M_{jk}&=&\sum_{i=1}^{N} m_{i}\left( \delta_{i,3j-2}\delta_{jk}+\delta_{i,3j-1}\delta_{jk}+  \delta_{i,3j}\delta_{jk}\right)\nonumber \\
M_{jk}&=&m_{3j-2}\delta_{jk}+m_{3j-1}\delta_{jk}+m_{3j}\delta_{jk} \nonumber \\
M_{jk}&=&\left( m_{3j-2}+m_{3j-1}+m_{3j} \right) \delta_{jk}
\end{eqnarray}
En \eqref{eq:11} debe resaltarse que para $j=k=3N$, el elemento de matriz $M_{3N,3N}$ requiere las masas $m_{3(3N)-2}=m_{9N-2}$, $m_{3(3N)-1}=m_{9N-1}$ y $m_{3(3N)}=m_{9N}$, sin embargo ! no hay $9N$ masas!, el n\'{u}mero de masas es el mismo n\'{u}mero de nodos: $N$, entonces, para poder calcular la matriz $\mathbf{M}$ es necesario definir lo siguiente:
\begin{equation}\label{eq:12}
m_{N+1},m_{N+2},...,m_{3N}=0
\end{equation}
Como a partir de $N+1$ las masas son nulas, la matriz de masa (que es diagonal) tiene elementos nulos si  
\begin{eqnarray*}
3j-2=N+1\\
j=\frac{N+3}{3} \\
\end{eqnarray*}
Como no siempre $N$ es m\'{u}ltiplo de 3, se escoge el entero menor que este m\'{a}s cerca al valor:
\begin{equation}\label{eq:13}
j=\left \lfloor\frac{N+3}{3}\right \rfloor
\end{equation}
Donde $\lfloor \rfloor$ representa la funci\'{o}n piso.

%\chapter{Anexo:}
\chapter{Anexo B: Fluctuaciones Pico}\label{AnexoB}
En las tablas \ref{tab:flu1} y \ref{tab:flu2} se muestran los identificadores de residuo que presentan m\'{a}ximos o m\'{i}nimos locales (picos) en las figuras \ref{fig:ANM_pre1} \ref{fig:ANM_pre2} y \ref{fig:ANM_pre3}, se muestra por cada distancia de corte entre $7\AA$ y $14\AA$ y para cada modelo de la estructura aplicado (Ca, Ca$+$Na, Ca$+$Na$+$GalCa$+$Gal). Los factores b est\'{a}n normalizados a 1. \\
\begin{table}[ht]
\centering
\begin{adjustbox}{width=1.0\textwidth,center}
 \begin{tabular}[c]{|c|c|}
\multicolumn{2}{c}{$R_c=$8$\AA$}\\\hline
\textbf{Residuo}&\textbf{Factor B}\\\hline
\multicolumn{2}{c}{Con solo C$\alpha$}\\\hline
       253& 0.0416425\\
       251& 0.0415383\\
       249& 0.0377258\\
       479& 0.0376493\\
       250& 0.0373515\\\hline
\multicolumn{2}{c}{C$\alpha$ $+$ Gal}\\\hline
       254& 0.0923907\\
       250& 0.0911848\\
       251& 0.0889088\\
       252& 0.0886782\\
       253& 0.0860388\\\hline
\multicolumn{2}{c}{C$\alpha$ $+$ Gal $+$ Na}\\\hline
       254& 0.0925526\\
       250& 0.0913644\\
       251& 0.0890739\\
       252& 0.0888302\\
       253& 0.0861924\\\hline
\multicolumn{2}{c}{C$\alpha$ $+$ Na}\\\hline
       480&  0.110012\\
        88&  0.109922\\
       479&  0.109521\\
       476&  0.107543\\
       438&  0.106579\\\hline
\end{tabular}
\begin{tabular}[c]{|c|c|}
\multicolumn{2}{c}{$R_c=$9$\AA$}\\\hline
\textbf{Residuo}&\textbf{Factor B}\\\hline
\multicolumn{2}{c}{Con solo C$\alpha$}\\\hline
       253& 0.0416425\\
       251& 0.0415383\\
       249& 0.0377258\\
       479& 0.0376493\\
       250& 0.0373515\\\hline
\multicolumn{2}{c}{C$\alpha$ $+$ Gal}\\\hline
       475&  0.105822\\
       479&  0.105115\\
       250&  0.103229\\
       252&  0.101724\\
        88&  0.100704\\\hline
\multicolumn{2}{c}{C$\alpha$ $+$ Gal $+$ Na}\\\hline
       475&  0.106073\\
       479&  0.105362\\
       250&  0.103468\\
       252&  0.101948\\
        88&  0.100918\\\hline
\multicolumn{2}{c}{C$\alpha$ $+$ Na}\\\hline
       255&  0.116419\\
       257&  0.114597\\
       249&  0.112165\\
       438&  0.112136\\
       258&  0.105612\\\hline
\end{tabular}
\begin{tabular}[c]{|c|c|}
\multicolumn{2}{c}{$R_c=$10$\AA$}\\\hline
\textbf{Residuo}&\textbf{Factor B}\\\hline
\multicolumn{2}{c}{Con solo C$\alpha$}\\\hline
       253&  0.134993\\
       257&  0.133941\\
       258&  0.133659\\
       254&  0.131689\\
       255&  0.130529\\\hline
\multicolumn{2}{c}{C$\alpha$ $+$ Gal}\\\hline
       256&  0.126339\\
       255&  0.125446\\
       257&  0.125102\\
       249&  0.123929\\
       258&  0.117366\\\hline
\multicolumn{2}{c}{C$\alpha$ $+$ Gal $+$ Na}\\\hline
       256&  0.126633\\
       255&  0.125735\\
       257&  0.125394\\
       249&  0.124238\\
       258&   0.11696\\\hline
\multicolumn{2}{c}{C$\alpha$ $+$ Na}\\\hline
       255&  0.134158\\
       257&  0.132059\\
       249&  0.129256\\
       438&  0.129223\\
       258&  0.121705\\\hline
\end{tabular}
\begin{tabular}[c]{|c|c|}
\multicolumn{2}{c}{$R_c=$11$\AA$}\\\hline
\textbf{Residuo}&\textbf{Factor B}\\\hline
\multicolumn{2}{c}{Con solo C$\alpha$}\\\hline
       479& 0.0732422\\
       253& 0.0729239\\
       250& 0.0727664\\
       478&  0.071105\\
       251& 0.0709591\\\hline
\multicolumn{2}{c}{C$\alpha$ $+$ Gal}\\\hline
       250&  0.130337\\
       256&  0.129298\\
       255&  0.129017\\
       258&   0.12784\\
       257&  0.127207\\\hline
\multicolumn{2}{c}{C$\alpha$ $+$ Gal $+$ Na}\\\hline
       259&  0.130679\\
       256&  0.129646\\
       255&  0.128714\\
       258&  0.127578\\
       257&  0.127547\\\hline
\multicolumn{2}{c}{C$\alpha$ $+$ Na}\\\hline
       256&  0.134707\\
       259&  0.134242\\
       255&  0.133804\\
       258&  0.130805\\
       257&  0.130507\\\hline
\end{tabular}
\begin{tabular}[c]{|c|c|}
\multicolumn{2}{c}{$R_c=$12$\AA$}\\\hline
\textbf{Residuo}&\textbf{Factor B}\\\hline
\multicolumn{2}{c}{Con solo C$\alpha$}\\\hline
       479& 0.0643314\\
       253& 0.0640519\\
       250& 0.0639135\\
       478& 0.0624543\\
       251&  0.062326\\\hline
\multicolumn{2}{c}{C$\alpha$ $+$ Gal}\\\hline
       254&  0.115136\\
       253&  0.114893\\
       259&  0.114486\\
       252&  0.111882\\
       256&  0.110051\\\hline
\multicolumn{2}{c}{C$\alpha$ $+$ Gal $+$ Na}\\\hline
       250& 0.0780249\\
       251& 0.0762417\\
       254& 0.0747905\\
       252& 0.0742422\\
       253& 0.0728088\\\hline
\multicolumn{2}{c}{C$\alpha$ $+$ Na}\\\hline
       258&  0.115959\\
       253&  0.113907\\
       255&   0.11262\\
       256&   0.11215\\
       252&  0.111072\\\hline
\end{tabular}
\begin{tabular}[c]{|c|c|}
\multicolumn{2}{c}{$R_c=$13$\AA$}\\\hline
\textbf{Residuo}&\textbf{Factor B}\\\hline
\multicolumn{2}{c}{Con solo C$\alpha$}\\\hline
       478& 0.0980013\\
       252& 0.0977737\\
       250& 0.0951496\\
       480& 0.0945263\\
       479& 0.0905842\\\hline
\multicolumn{2}{c}{C$\alpha$ $+$ Gal}\\\hline
       258&    0.1166\\
       253&  0.114328\\
       255&  0.112683\\
       256&  0.112637\\
       252&  0.111443\\\hline
\multicolumn{2}{c}{C$\alpha$ $+$ Gal $+$ Na}\\\hline
       251& 0.0895345\\
       250& 0.0884175\\
       254& 0.0862215\\
       253& 0.0858509\\
       252& 0.0829043\\\hline
\multicolumn{2}{c}{C$\alpha$ $+$ Na}\\\hline
       256&  0.117632\\
       258&  0.117381\\
       255&  0.117372\\
       252&  0.116511\\
       253&   0.11622\\\hline
\end{tabular}
\begin{tabular}[c]{|c|c|}
\multicolumn{2}{c}{$R_c=$14$\AA$}\\\hline
\textbf{Residuo}&\textbf{Factor B}\\\hline
\multicolumn{2}{c}{Con solo C$\alpha$}\\\hline
       250&   0.10348\\
       476&  0.102362\\
       480&  0.101862\\
       253&  0.101347\\
       479& 0.0985975\\\hline
\multicolumn{2}{c}{C$\alpha$ $+$ Gal}\\\hline
       256&   0.11746\\
       258&   0.11726\\
       255&  0.117135\\
       252&  0.116212\\
       253&   0.11608\\\hline
\multicolumn{2}{c}{C$\alpha$ $+$ Gal $+$ Na}\\\hline
       251& 0.0895255\\
       250& 0.0884151\\
       254& 0.0862144\\
       253& 0.0858443\\
       252& 0.0828968\\\hline
\multicolumn{2}{c}{C$\alpha$ $+$ Na}\\\hline
       256&  0.117624\\
       258&  0.117373\\
       255&  0.117364\\
       252&  0.116503\\
       253&  0.116212\\\hline
\end{tabular}

 \end{adjustbox}
 \caption{Lista de los cinco n\'{u}meros de residuo que corresponden a los menores factores B}\label{tab:flu1}
\end{table}
\begin{table}[H]
\centering
\begin{adjustbox}{width=1.0\textwidth,center}
 \begin{tabular}[c]{|c|c|}
\multicolumn{2}{c}{$R_c=$8$\AA$}\\\hline
\textbf{Residuo}&\textbf{Factor B}\\\hline
\multicolumn{2}{c}{Con solo C$\alpha$}\\\hline
       543&  0.999999\\
       318&  0.798209\\
       319&   0.77496\\
       321&  0.741277\\
       317&  0.668767\\
\hline
\multicolumn{2}{c}{C$\alpha$ $+$ Gal}\\\hline
       209&  0.925647\\
       210&  0.879899\\
       318&  0.781582\\
       386&  0.775865\\
       384&  0.773399\\
\hline
\multicolumn{2}{c}{C$\alpha$ $+$ Gal $+$ Na}\\\hline
       209&  0.927167\\
       210&  0.881346\\
       318&  0.782798\\
       386&   0.77716\\
       384&  0.774713\\
\hline
\multicolumn{2}{c}{C$\alpha$ $+$ Na}\\\hline
       209&  0.868804\\
       210&  0.845608\\
       318&  0.827396\\
       211&  0.716527\\
       384&  0.702102\\
\hline
\end{tabular}
\begin{tabular}[c]{|c|c|}
\multicolumn{2}{c}{$R_c=$9$\AA$}\\\hline
\textbf{Residuo}&\textbf{Factor B}\\\hline
\multicolumn{2}{c}{Con solo C$\alpha$}\\\hline
       543&  0.999999\\
       318&  0.798209\\
       319&   0.77496\\
       321&  0.741277\\
       317&  0.668767\\
\hline
\multicolumn{2}{c}{C$\alpha$ $+$ Gal}\\\hline
       209&  0.882793\\
       210&  0.859023\\
       318&   0.84161\\
       211&  0.727772\\
       384&  0.714453\\
\hline
\multicolumn{2}{c}{C$\alpha$ $+$ Gal $+$ Na}\\\hline
       209&  0.884548\\
       210&  0.860736\\
       318&  0.843289\\
       211&  0.729196\\
       384&  0.715938\\
\hline
\multicolumn{2}{c}{C$\alpha$ $+$ Na}\\\hline
       210&  0.851038\\
       385&   0.83699\\
       209&  0.831604\\
       318&  0.789342\\
       211&  0.702455\\
\hline
\end{tabular}
\begin{tabular}[c]{|c|c|}
\multicolumn{2}{c}{$R_c=$10$\AA$}\\\hline
\textbf{Residuo}&\textbf{Factor B}\\\hline
\multicolumn{2}{c}{Con solo C$\alpha$}\\\hline
       209&   0.97897\\
       210&  0.971493\\
       318&  0.781321\\
       311&  0.755165\\
       315&  0.744612\\
\hline
\multicolumn{2}{c}{C$\alpha$ $+$ Gal}\\\hline
       210&  0.997761\\
       385&  0.982432\\
       209&  0.974967\\
       318&  0.926077\\
       211&  0.823312\\
\hline
\multicolumn{2}{c}{C$\alpha$ $+$ Gal $+$ Na}\\\hline
       210&  0.999999\\
       385&  0.984738\\
       209&  0.977157\\
       318&  0.928175\\
       211&  0.825091\\
\hline
\multicolumn{2}{c}{C$\alpha$ $+$ Na}\\\hline
       210&  0.980716\\
       385&  0.964528\\
       209&  0.958321\\
       318&  0.909619\\
       211&  0.809492\\
\hline
\end{tabular}
\begin{tabular}[c]{|c|c|}
\multicolumn{2}{c}{$R_c=$11$\AA$}\\\hline
\textbf{Residuo}&\textbf{Factor B}\\\hline
\multicolumn{2}{c}{Con solo C$\alpha$}\\\hline
       209&  0.857345\\
       386&  0.806943\\
       318&  0.796594\\
       388&  0.790143\\
       384&  0.737336\\
\hline
\multicolumn{2}{c}{C$\alpha$ $+$ Gal}\\\hline
       210&  0.997539\\
       209&  0.923375\\
       318&  0.866003\\
       382&  0.813186\\
       211&  0.803287\\
\hline
\multicolumn{2}{c}{C$\alpha$ $+$ Gal $+$ Na}\\\hline
       210&         1\\
       209&  0.925618\\
       318&  0.868137\\
       382&  0.815369\\
       211&  0.805206\\
\hline
\multicolumn{2}{c}{C$\alpha$ $+$ Na}\\\hline
       210&  0.978538\\
       209&   0.90573\\
       318&  0.848982\\
       382&  0.797229\\
       211&  0.788358\\
\hline
\end{tabular}
\begin{tabular}[c]{|c|c|}
\multicolumn{2}{c}{$R_c=$12$\AA$}\\\hline
\textbf{Residuo}&\textbf{Factor B}\\\hline
\multicolumn{2}{c}{Con solo C$\alpha$}\\\hline
       209&  0.753039\\
       386&  0.708769\\
       318&  0.699679\\
       388&  0.694012\\
       384&   0.64763\\
\hline
\multicolumn{2}{c}{C$\alpha$ $+$ Gal}\\\hline
       210&  0.877561\\
       209&  0.857573\\
       318&  0.737105\\
       211&  0.700336\\
       545&  0.691128\\
\hline
\multicolumn{2}{c}{C$\alpha$ $+$ Gal $+$ Na}\\\hline
       209&  0.999999\\
       318&  0.895038\\
       388&    0.7751\\
       343&  0.764006\\
       210&  0.752476\\
\hline
\multicolumn{2}{c}{C$\alpha$ $+$ Na}\\\hline
       210&   0.94773\\
       209&  0.821616\\
       318&  0.695025\\
       211&  0.676508\\
       545&  0.615122\\
\hline
\end{tabular}
\begin{tabular}[c]{|c|c|}
\multicolumn{2}{c}{$R_c=$13$\AA$}\\\hline
\textbf{Residuo}&\textbf{Factor B}\\\hline
\multicolumn{2}{c}{Con solo C$\alpha$}\\\hline
       209&  0.981844\\
       318&   0.89057\\
       210&  0.832029\\
       317&  0.726279\\
       314&  0.710232\\
\hline
\multicolumn{2}{c}{C$\alpha$ $+$ Gal}\\\hline
       210&  0.978828\\
       209&  0.848552\\
       318&  0.717871\\
       211&  0.698532\\
       545&  0.641043\\
\hline
\multicolumn{2}{c}{C$\alpha$ $+$ Gal $+$ Na}\\\hline
       209&  0.999999\\
       318&  0.907254\\
       210&  0.847078\\
       317&  0.739736\\
       314&  0.723279\\
\hline
\multicolumn{2}{c}{C$\alpha$ $+$ Na}\\\hline
       209&  0.889031\\
       210&  0.874584\\
       318&  0.696198\\
       211&  0.662066\\
       311&  0.656291\\
\hline
\end{tabular}
\begin{tabular}[c]{|c|c|}
\multicolumn{2}{c}{$R_c=$14$\AA$}\\\hline
\textbf{Residuo}&\textbf{Factor B}\\\hline
\multicolumn{2}{c}{Con solo C$\alpha$}\\\hline
       209&  0.947309\\
       210&  0.900641\\
       318&  0.799395\\
       386&  0.792957\\
       384&   0.79044\\
\hline
\multicolumn{2}{c}{C$\alpha$ $+$ Gal}\\\hline
       209&  0.908923\\
       210&  0.894141\\
       318&  0.711833\\
       211&  0.676765\\
       311&  0.670858\\
\hline
\multicolumn{2}{c}{C$\alpha$ $+$ Gal $+$ Na}\\\hline
       209&         1\\
       318&   0.90724\\
       210&   0.84708\\
       317&  0.739725\\
       314&  0.723268\\
\hline
\multicolumn{2}{c}{C$\alpha$ $+$ Na}\\\hline
       209&  0.888968\\
       210&  0.874523\\
       318&  0.696149\\
       211&   0.66202\\
       311&  0.656245\\
\hline
\end{tabular}

 \end{adjustbox}
  \caption{Lista de los cinco n\'{u}meros de residuo que corresponden a los mayores factores B}\label{tab:flu2}
\end{table}

\end{appendix}
