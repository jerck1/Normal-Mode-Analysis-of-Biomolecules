\chapter*{Lista de s\'{\i}mbolos}
\addcontentsline{toc}{chapter}{\numberline{}Lista de s\'{\i}mbolos}
\section*{S\'{\i}mbolos con letras latinas}
 \label{simbolos}
 %\renewcommand{\arraystretch}{3}
\begin{longtable}{p{2cm}p{4cm}p{2cm}p{8cm}}
\textbf{S\'{\i}mbolo}&\textbf{T\'{e}rmino}&\textbf{Unidad SI}&\textbf{Definici\'{o}n}\\[0.5ex]\hline
\endfirsthead%
\textbf{S\'{\i}mbolo}&\textbf{T\'{e}rmino}&\textbf{Unidad SI}&\textbf{Definici\'{o}n}\\[0.5ex]\hline
\endhead%
      $Z$&N\'{u}mero at\'{o}mico&\hspace{6pt}1&N\'{u}mero de protones en un \'{a}tomo\\%
      pH &Potencial de hidr\'{o}geno &\hspace{6pt}1&Grado de acidez o alcanilidad en una soluci\'{o}n\\%
      $r_{ij}$&Distancia entre part\'{i}culas $i$ y $j$&\hspace{6pt}m&Secci\'{o}n \ref{ssec:Estruc}\\%
      e&Carga del electr\'{o}n&\hspace{6pt}A.s&$e=1.602\times 10^{19}$C\\%
\ce{P_{1}}&Sistema tricl\'{i}nico&\hspace{6pt}-&Secci\'{o}n \ref{sec:esvSGLT}\\%
\ce{P_{21}}&Sistema monocl\'{i}nico&\hspace{6pt}-&Secci\'{o}n \ref{sec:esvSGLT}\\
\ce{P_{21}P_{21}P_{21}}&Sistema ortorr\'{o}mbico&\hspace{6pt}-&Secci\'{o}n \ref{sec:esvSGLT}\\
$V$ &Energ\'{i}a potencial&\hspace{6pt}J&Forma de energ\'{i}a que depende de la posici\'{o}n o una propiedad del objeto\\%
$T$&Energ\'{i}a cin\'{e}tica&\hspace{6pt}J&Forma de energ\'{i}a que depende de la velocidad de un objeto\\%
$E$  &Energ\'{i}a total&\hspace{6pt}J&Suma de las formas de energ\'{i}a\\%
$q$&Coordenada generalizada&\hspace{6pt}-&Par\'{a}metro que representa la configuraci\'{o}n del sistema respecto a una referencia.\\%
$N$&N\'{u}mero de part\'{i}culas&\hspace{6pt}1&N\'{u}mero de part\'{i}culas\\%
$\frac{\partial}{\partial q_i}$&Derivada parcial&\hspace{6pt}-&Raz\'{o}n de cambio instant\'{a}nea respecto a la coordenada $q_i$\\%
$\mathbf{A}$&Matriz A (en negrita)&\hspace{6pt}-&$\mathbf{A}=\{a_{ij}\}$\\%
$\mathbf{r_i}$&Vector de posici\'{o}n part\'{i}cula $i$&\hspace{6pt}m&$\mathbf{r_i}=\left(x_i,y_i,z_i\right)$\\%
$\mathbf{M}$&Matriz de masa del sistema&\hspace{6pt}kg&Secci\'{o}n \ref{sec:MecNMA}\\%
$\mathbf{U}$&Matriz de constantes el\'{a}sticas&\hspace{6pt}N/m&Ecuaci\'{o}n \eqref{eq:5}\\%
$\mathbf{r}$&Coordenadas de masa ponderada para la posici\'{o}n&\hspace{6pt}$\mathrm{m.kg^{1/2}}$ &Ecuaciones \eqref{eq:15}\\%
$\mathbf{K}$&Constantes el\'{a}sticas en las coordenadas ponderadas&\hspace{6pt}$\mathrm{N/kg.m}$ &Ecuaciones \eqref{eq:15}\\%
$\lambda_k$&Autovalor $k$\'{e}simo &\hspace{6pt}$\mathrm{rad^2/s^2}$&Ecuaci\'{o}n \eqref{eq:18}\\%
$\mathbf{a_k}$&Autovector correspondiente a $\lambda_k$&\hspace{6pt}$\mathrm{m.kg^{1/2}}$ &Ecuaci\'{o}n \eqref{eq:18}\\%
$\zeta_k$   &Coordenada normal correspondiente a $\omega_k$&\hspace{6pt}1 &Ecuaci\'{o}n \eqref{eq:nor}\\%
$E_s$&Energ\'{i}a del sistema &\hspace{6pt}J  &Suma de las formas de energ\'{i}a\\%
$k_B$&Constante de Boltzmann &\hspace{6pt}J/K&$k_B=1.38\times 10^{-23}$J/K\\%
$p$&Densidad de probabilidad &\hspace{6pt}1 &Probabilidad por unidad de volumen del espacio de fase\\%
$Z$  &Funci\'{o}n de partici\'{o}n&\hspace{6pt}1 &Probabilidad por unidad de volumen del espacio de fase\\%
$T$ &Temperatura &\hspace{6pt}K &Es una mediad objetiva de que tan caliente o fr\'{i}o\\%
$H(\mathbf{p},\mathbf{q})$ &Hamiltoniano  &\hspace{6pt}J &Transformada de Legendre del Lagrangiano\\%
$n$ &N\'{u}mero de coordenadas    &\hspace{6pt}1 &$n=3N$\\%
$n$&N\'{u}mero de coordenadas   &\hspace{6pt}1 &$n=3N$\\%
$\prod_i^m x_i$&Productoria&\hspace{6pt}-&$\prod_i^m x_i=x_1x_2...x_m$\\%
$\hbar$&Constante de Plank reducida &\hspace{6pt}J.s &$\hbar=\frac{h}{2\pi}=1.054\times 10^{-34}$J.s\\%
$R_{ij}^0$&Distancia de equilibrio entre los nodos $i$ y $j$ &\hspace{6pt}m&Secci\'{o}n \ref{ssec:ANM}\\%
$R_{ij}$&Distancia entre los nodos $i$ y $j$ &\hspace{6pt}m&Secci\'{o}n \ref{ssec:ANM}\\%
\end{longtable}
\vspace{5ex}
\section*{S\'{\i}mbolos con letras griegas}

\begin{longtable}{p{2cm}p{3.5cm}p{2cm}p{8cm}}
\textbf{S\'{\i}mbolo}&\textbf{T\'{e}rmino}&\textbf{Unidad SI}&\textbf{Definici\'{o}n}\\[0.5ex] \hline%
\endfirsthead%
\textbf{S\'{\i}mbolo}&\textbf{T\'{e}rmino}&\textbf{Unidad SI}&\textbf{Definici\'{o}n}\\[0.5ex] \hline%
\endhead%
\renewcommand{\arraystretch}{1.3}
 \label{simbolosg}
 $(\phi,\psi,\omega)$&\'{A}ngulos diedros&\hspace{6pt}rad &Secci\'{o}n \ref{ssec:Estruc} \\
 $\alpha$ &$\alpha$ h\'{e}lice&\hspace{6pt}- &Secci\'{o}n \ref{ssec:Estruc}\\
     $\alpha_R$&$\alpha$ h\'{e}lice derecha&\hspace{6pt}- &Secci\'{o}n \ref{ssec:Estruc}\\
     $\alpha_L$&$\alpha$ h\'{e}lice izquierda&\hspace{6pt}- &Secci\'{o}n \ref{ssec:Estruc} \\
  $\beta$&Hoja $\beta$&\hspace{6pt}- &Secci\'{o}n \ref{ssec:Estruc} \\
  $\epsilon_0$&Permitividad en el vac\'{i}o&\hspace{6pt}$\mathrm{C^2/N.m^2}$&$\epsilon_0=8.85\times 10^{-12}\mathrm{C^2/N.m^2}$ \\ 
   $\epsilon_r$&Permitividad relativa&\hspace{6pt}1&Secci\'{o}n \ref{ssec:Estruc} \\
   $\gamma_{ij}$&Constante el\'{a}stica entre los nodos $i$ y $j$&\hspace{6pt}N/m&Secci\'{o}n \ref{ssec:ANM} \\
   $\omega$&Frecuencia angular&\hspace{6pt}$\mathrm{rad/s}$&Raz\'{o}n de cambio del \'{a}ngulo en el tiempo. Secci\'{o}n \ref{sec:MecNMA}\\%
     \hline
\end{longtable}


\section*{Abreviaturas}
\begin{longtable}[l]{lll}\hline
   \textbf{Abreviatura} & \textbf{Definici\'{o}n} \\
 \hline%
  \endfirsthead%
 \textbf{Abreviatura} & \textbf{Definici\'{o}n} \\
  \hline%
 \endhead%
\renewcommand{\arraystretch}{1.4}\label{abre}
PDB&Protein Data Bank\\
ANM&AModelo de Redes Anisotr\'{o}picas\\
GNM&Modelo de Redes Gaussianas\\
vSLGT&Cotransportador de Sodio/Galactosa de la bacteria \textit{Vibrio Parahemoliticus}\\
SSS&Familia de simportadores de Sodio/Soluto\\
TM&Segmento Transmembranal\\ 
hBRP2&Human Retinol Binding Protein 2\\
ATP&Adenos\'{i}n Trifosfato\\
SGLT1&Transportador de sodio/glucosa 1\\
SGLT2&Transportador de sodio/glucosa 2\\
NIS&Simportador de sodio/yoduro\\
LeuT&Transportador de Leucina\\
APC&Superfamilia amino\'{a}cido/poliamina oragnocati\'{o}n\\
NSS&Familia de simportadores de sodio/neurotransmisor \\
MD& Din\'{a}mica Molecular\\
BPTI&Inhibidor de la Tripsina Pancre\'{a}tica Bovina\\
GltPh&Transportador de Glutamato hom\'{o}logo\\
MAVEN& An\'{a}lisis de Movimiento y Visualizaci\'{o}n de Redes El\'{a}sticas\\
NMA&An\'{a}lis de Modos Normales\\
ENM&Modelos de Redes El\'{a}sticas\\
PCA&An\'{a}lisis por Componentes Principales\\
BNM&Modelos de Bloque R\'{i}gido\\
RTB&Rotaciones y Traslaciones de Bloques\\
OPM&Base de datos Orientations in Transmembrane Proteins\\
\hline
\end{longtable}
\subsection*{20 Amino\'{a}cidos Com\'{u}nes}
  \begin{longtable}[l]{lll}
   \textbf{Amino\'{a}cido} & \multicolumn{2}{l}{\textbf{Abreviatura}} \\
  \cline{2-3}
  &\textbf{3 letras}&\textbf{1 letra}\\[0.5ex] \hline%
  \endfirsthead%
 \textbf{Amino\'{a}cido} & \multicolumn{2}{l}{\textbf{Abreviatura}} \\
  \cline{2-3}
  &\textbf{3 letras}&\textbf{1 letra}\\[0.5ex] \hline%
 \endhead%
\renewcommand{\arraystretch}{1.4}\label{amino}
Alanina&Ala&A\\
Arginina&Arg&R\\
Asparagina&Asn&N\\
Aspartato&Asp&D\\
Ciste\'{i}na&Cys&C\\
Glutamato&Glu&E\\
Glutamina&Gln&Q\\
Glicina&Gly&G\\
%\columnbreak
Histidina&His&H\\
Isoleucina&Ile&I\\
Leucina&Leu&L\\
Lisina&Lys&K\\
Metionina&Met&M\\
Fenilalanina&Phe&F\\
Prolina&Pro&P\\
Serina&Ser&S\\
Treonina&Thr&T\\
Tript\'{o}fano&Trp&W\\
Tirosina&Tyr&Y\\
Valina&Val&V\\ \hline
\end{longtable}
\newpage
\section*{Sub\'{\i}ndices}
\begin{longtable}{ll}
  \textbf{Sub\'{\i}ndice} & \textbf{T\'{e}rmino} \\[0.5ex] \hline%
  \endfirsthead%
  \textbf{Sub\'{\i}ndice} & \textbf{T\'{e}rmino} \\[0.5ex] \hline%
  \endhead%
\renewcommand{\arraystretch}{1.4}\label{simbolosg}

 $i$&Part\'{i}cula $i$\\%
 $j$&Part\'{i}cula $j$\\%
 $k$&Part\'{i}cula $k$\\%


\end{longtable}
\setlength{\extrarowheight}{0pt}